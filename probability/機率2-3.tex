\documentclass{article}
\usepackage[encapsulated]{CJK}
\linespread{1.36}
\voffset=-1in
\begin{document}
%中文時不能用 \fontsize{12pt}{20pt}\selectfont
\fontsize{13pt}{20pt}\selectfont

\begin{CJK}{UTF8}{bsmi} % 開始 CJK 環境,設定編碼,設定字體
\begin{tabular}[t]{|l|l|l|}
\hline


A & B & C\\
\hline
D & E & F\\
\hline
G & H & I \\
\hline
\end{tabular} 
\\


由於其對稱性, A = I, B = F, D = H

\noindent
E(A$\rightarrow$C) = 1 + $\frac{1}{2}$E(D$\rightarrow$C) + $\frac{1}{2}$E(B$\rightarrow$C)

= 1 + $\frac{1}{2}(\frac{1}{2} \times$ 0+ ($\frac{1}{2}$ + $\frac{1}{2}E(A\rightarrow C))) + \frac{1}{2}E(B \rightarrow$C)

= $\frac{5}{4} + \frac{1}{4}E(A\rightarrow C)) + \frac{1}{2}(1+\frac{1}{2}E(A\rightarrow C) + \frac{1}{2}E(G\rightarrow C))$\\
E(A$\rightarrow$C) = $\frac{7}{2} + \frac{1}{2}E(G\rightarrow C)$

\noindent
E(G$\rightarrow$C) = 1 + E(B$\rightarrow$C)
= 2 + $\frac{1}{2}$E(A$\rightarrow$C) + $\frac{1}{2}$E(G$\rightarrow$C)\\
E(G$\rightarrow$C) = 15
\end{CJK} % 結束 CJK 環境
\end{document}


