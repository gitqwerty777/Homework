\documentclass{article}
\begin{document}
\fontsize{12pt}{20pt}\selectfont
Exist $c_{1}, c_{2},$ and $n_{0}$ that
 $ \Theta(g(n)) = \{f(n): 0 \leq c_{1}g(n) \leq f(n) \leq c_{2}g(n)\}$
  for all $n \geq n_{0} $
  
  
Exist $c_{1}, c_{2},$ and $n_{0}$ that
 $ o(g(n)) = \{f(n): 0 \leq f(n) < cg(n)\}$
  for all $n \geq n_{0} $
  
  
For Example: $2n^{2} = O(n^{2})$, but $2n^{2} \neq o(n^{2})$.


$f(n) = \omega(g(n))$ if and only if $g(n) = o(f(n))$

Because $\Theta(g(n))$ is a set, $ f(n) \in \Theta(g(n))$ , but we always write $f(n) = \Theta(g(n))$.
\\Property:
\\ \indent 1.Transitivity: $f(n) = \Theta(g(n))$ and $g(n) = \Theta(h(n))$
then $f(n) = \Theta(h(n))$
\\ \indent 2.Reflexivity: $f(n) = \Theta(f(n))$
\\ \indent 3.Symmetry: $f(n) = \Theta(g(n))$ if and only if $g(n) = \Theta(f(n))$
\\ \indent 4.Transpose Symmetry: $f(n) = \Theta(g(n))$ if and only if $g(n) = \Omega(f(n))$
\\ $f(n) = o(g(n))$ if and only if $g(n) = \omega(f(n))$

$e^{x}= 1+x+\frac{x^{2}}{2!}+\frac{x^{3}}{3!} ... = \sum_{i=0}^{\infty}\frac{x^{i}}{i!}$

$1+x \leq e^{x} \leq 1+x+x^{2}$

Fibonacci: $F_{i} = \frac{\rho^{i} - \bar{\rho^{i}}}{\sqrt{5}}$
\\$\rho$ is golden ratio
$F_{i}$ equal to $\frac{\rho^{i}}{\sqrt{5}}$ rounded to the nearist integer

\end{document}
