\documentclass{article}

\usepackage{fancyhdr}
\usepackage{extramarks}
\usepackage{amsmath}
\usepackage{amsthm}
\usepackage{amsfonts}
\usepackage{tikz}
\usepackage[plain]{algorithm}
\usepackage{algpseudocode}
\usepackage[encapsulated]{CJK}
\usepackage{graphicx}
\usepackage{caption}
\usepackage{subcaption}
\graphicspath{ {./images/} }

\usetikzlibrary{automata,positioning}

%
% Basic Document Settings
%

\topmargin=-0.45in
\evensidemargin=0in
\oddsidemargin=0in
\textwidth=6.5in
\textheight=9.0in
\headsep=0.25in

\linespread{1.1}

\pagestyle{fancy}
\lhead{\hmwkAuthorName}
\chead{\hmwkClass\:\hmwkTitle}
\rhead{\firstxmark}
\lfoot{\lastxmark}
\cfoot{\thepage}

\renewcommand\headrulewidth{0.4pt}
\renewcommand\footrulewidth{0.4pt}

\setlength\parindent{0pt}

%
% Create Problem Sections
%

\newcommand{\enterProblemHeader}[1]{
    \nobreak\extramarks{}{Problem \arabic{#1} continued on next page\ldots}\nobreak{}
    \nobreak\extramarks{Problem \arabic{#1} (continued)}{Problem \arabic{#1} continued on next page\ldots}\nobreak{}
}

\newcommand{\exitProblemHeader}[1]{
    \nobreak\extramarks{Problem \arabic{#1} (continued)}{Problem \arabic{#1} continued on next page\ldots}\nobreak{}
    \stepcounter{#1}
    \nobreak\extramarks{Problem \arabic{#1}}{}\nobreak{}
}

\setcounter{secnumdepth}{0}
\newcounter{partCounter}
\newcounter{homeworkProblemCounter}
\setcounter{homeworkProblemCounter}{1}
\nobreak\extramarks{Problem \arabic{homeworkProblemCounter}}{}\nobreak{}

%
% Homework Problem Environment
%
% This environment takes an optional argument. When given, it will adjust the
% problem counter. This is useful for when the problems given for your
% assignment aren't sequential. See the last 3 problems of this template for an
% example.
%
\newenvironment{homeworkProblem}[1][-1]{
    \ifnum#1>0
        \setcounter{homeworkProblemCounter}{#1}
    \fi
    \section{Problem \arabic{homeworkProblemCounter}}
    \setcounter{partCounter}{1}
    \enterProblemHeader{homeworkProblemCounter}
}{
    \exitProblemHeader{homeworkProblemCounter}
}

%
% Homework Details
%   - Title
%   - Due date
%   - Class
%   - Section/Time
%   - Instructor
%   - Author
%

\newcommand{\hmwkTitle}{Homework\ \#4}
%\newcommand{\hmwkDueDate}{September 17, 2015}
\newcommand{\hmwkClass}{Graph Theory}
\newcommand{\hmwkClassTime}{}
\newcommand{\hmwkClassInstructor}{}
\newcommand{\hmwkAuthorName}{Lin Hung Cheng B01902059}

%
% Title Page
%

\title{
    \vspace{2in}
    \textmd{\textbf{\hmwkClass:\ \hmwkTitle}}\\
    %\normalsize\vspace{0.1in}\small{Due\ on\ \hmwkDueDate\ at 3:10pm}\\
    %\vspace{0.1in}\large{\textit{\hmwkClassInstructor\ \hmwkClassTime}}
    \vspace{3in}
}

\author{\textbf{\hmwkAuthorName}}
\date{}

\renewcommand{\part}[1]{\textbf{\large Part \Alph{partCounter}}\stepcounter{partCounter}\\}

%
% Various Helper Commands
%

% Useful for algorithms
\newcommand{\alg}[1]{\textsc{\bfseries \footnotesize #1}}

% For derivatives
\newcommand{\deriv}[1]{\frac{\mathrm{d}}{\mathrm{d}x} (#1)}

% For partial derivatives
\newcommand{\pderiv}[2]{\frac{\partial}{\partial #1} (#2)}

% Integral dx
\newcommand{\dx}{\mathrm{d}x}

% Alias for the Solution section header
\newcommand{\solution}{\textbf{\large Solution}}

% Probability commands: Expectation, Variance, Covariance, Bias
\newcommand{\E}{\mathrm{E}}
\newcommand{\Var}{\mathrm{Var}}
\newcommand{\Cov}{\mathrm{Cov}}
\newcommand{\Bias}{\mathrm{Bias}}

\begin{document}

\maketitle

\pagebreak

\begin{homeworkProblem}
  \begin{CJK}{UTF8}{bsmi} % 開始 CJK
    在第2.5 節Euler 迴路的案例中, 改用連續空間表示圖2.10 的圖、但同時要含SAME 和MARK 兩個欄位。
    
    \solution

    說明: First element of DATA is index 1. Elements in DATA: [data, same, mark]\\
    BEG: [1, 5, 11, 15]\\
    DATA: [[4, 18, 0], [2, 8, 0], [2, 9, 0], [2, 10, 0], [4, 17, 0], [3, 13, 0], [3, 14, 0], [1, 2, 0], [1, 3, 0], [1, 4, 0], [4, 15, 0], [4, 16, 0], [2, 6, 0], [2, 7, 0], [3, 11, 0], [3, 12, 0], [2, 5, 0], [1, 1, 0]]\\
    
  \end{CJK}
\end{homeworkProblem}

\begin{homeworkProblem}
  \begin{CJK}{UTF8}{bsmi} % 開始 CJK
    在第2.5 節Euler迴路的案例中,利用連續空間表示圖、據以改寫程式inputgraph。

    \begin{algorithm}[]
      \begin{algorithmic}[1]
        \Function{INPUTGRAPH}{$V, E$}   \Comment{array index start from 1}
        \State{} $DATA \gets array(E.length*2)$ \Comment{use DATA to represent [data, same, mark]}
        \For{each element d in DATA}
        \State $d.mark \gets 0$
        \State $d.same \gets 0$
        \EndFor
        \State{} $BEG \gets array(E.length*2)$
        \For{edge in E}
        \State{} $v[edge.first].push(edge.second)$ \Comment{first and second are two vertex in the edge}
        \State{} $v[edge.second].push(edge.first)$
        \EndFor
        \State{} $BEG[i] \gets 1$
        \For{$i \gets 2, n$}
        \State{} $BEG[i] \gets BEG[i-1]+v[i-1].length$
        \EndFor
        \State{} $index \gets 1$

        \For{$i \gets 1, n$}
        \For{each node N in v[i]}
        \State{} $DATA[index].data \gets N$
        \State{} $index \gets index + 1$
        \EndFor
        \EndFor

        \For{$i \gets 1, n$}
        \For{$j \gets 1, v[i].length$}
        \If{$v[i][j].same = 0$}
        \State $sameIndex \gets findSameIndex(v[i][j].data, i)$
        \State \Comment{findSameIndex: goto BEG=v[i][j].data and return the nearest index which same = 0 and data = i}
        \State $myIndex \gets BEG[i]+j-1$
        \State $DATA[sameIndex].same \gets myIndex$
        \State $DATA[myIndex].same \gets sameIndex$
        \EndIf
        \EndFor
        \EndFor
        
        \State \Return{BEG, DATA}
        \EndFunction
      \end{algorithmic}
      \caption{inputGraph}
    \end{algorithm}

\end{CJK} % 結束 CJK 環境 
\end{homeworkProblem}

\pagebreak

\begin{homeworkProblem}
  \begin{CJK}{UTF8}{bsmi} % 開始 CJK
    在第2.5 節Euler 迴路的案例中, 利用連續空間表示圖、據以改寫程式Eulertour。

    \begin{algorithm}[]
      \begin{algorithmic}[1]
        \Function{EULERTOUR}{$V, E, BEG, DATA$}
        \State $tour \gets array(E.length)$
        \State $tournext \gets array(E.length)$
        \State $tourprev \gets array(E.length)$
        \State $new \gets 1$
        \State $cur \gets new$
        \State $tour[new] \gets 1$
        \State $tournext[new] \gets 0$
        \State $tourprev[new] \gets 0$
        \While{$cur \neq 0$}
        \State $i \gets tour[cur]$
        \If{$BEG[i] \neq 0$} \Comment{BEG has the same usage as ADJ}
        \State $new \gets new+1$
        \State $tour[new] \gets data[BEG[i]]$
        \State $tournext[new] \gets tournext[cur]$
        \State $tourprev[new] \gets cur$
        \State $cur \gets new$
        \State $mark[BEG[i]] \gets 1$
        \State $mark[DATA[BEG[i]].same] \gets 1$
        \State $j \gets DATA[BEG[i]].next$
        \While{$j \neq 0$ and $mark[j] = 1$}
        \State $j \gets DATA[j].next$
        \EndWhile
        \Else
        \State $cur \gets tourprev[cur]$
        \EndIf
        \EndWhile
        \State \Return{tour, tourprev, tournext}
        \EndFunction
      \end{algorithmic}
      \caption{Euler Tour}
    \end{algorithm}
    
  \end{CJK} % 結束 CJK 環境 
\end{homeworkProblem}

\begin{homeworkProblem}
  \begin{CJK}{UTF8}{bsmi} % 開始 CJK
    在聯集尋找問題中, 我們若採用NAME、集合表列與SIZE 的方法, 試證明,若演算法總共做了n-1 次聯集的運算, 那需時會是O(n log n)

    \solution
    
    考慮最差情況(使計算量最多),最後一次運算應為$\frac{n}{2}$ 和$\frac{n}{2}$ 的聯集;而要形成$\frac{n}{2}$,最差的情況是 $\frac{n}{4}$ 和 $\frac{n}{4}$ 的聯集……,以此類推。\\
    直到 1 和 1的聯集,可知$\frac{n}{2}$最多出現1次,$\frac{n}{4}$最多出現2次...,最多共需要$\frac{n}{2} * 1 + \frac{n}{4} * 2 + \frac{n}{8} * 4 + ... + 1 * \frac{n}{2}$次計算,因為此式共有$\log_2{n}$項,每項和均為$\frac{n}{2}$,所以其和$\frac{n}{2}\log_2{n} \leq n\log_2{n}$
  \end{CJK} % 結束 CJK 環境    
\end{homeworkProblem}

\end{document}

%%% Local Variables:
%%% mode: latex
%%% TeX-master: t
%%% End:
