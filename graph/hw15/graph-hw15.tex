\documentclass{article}

\usepackage{fancyhdr}
\usepackage{extramarks}
\usepackage{amsmath}
\usepackage{amsthm}
\usepackage{amsfonts}
\usepackage{tikz}
\usepackage[plain]{algorithm}
\usepackage{algpseudocode}
\usepackage[encapsulated]{CJK}
\usepackage{graphicx}
\usepackage{caption}
\usepackage{subcaption}
\usepackage{graphics, tikz, tkz-berge, tkz-graph}
\graphicspath{ {./images/} }

\usetikzlibrary{automata,positioning}

%
% Basic Document Settings
%

\topmargin=-0.45in
\evensidemargin=0in
\oddsidemargin=0in
\textwidth=6.5in
\textheight=9.0in
\headsep=0.25in

\linespread{1.1}

\pagestyle{fancy}
\lhead{\hmwkAuthorName}
\chead{\hmwkClass\:\hmwkTitle}
\rhead{\firstxmark}
\lfoot{\lastxmark}
\cfoot{\thepage}

\renewcommand\headrulewidth{0.4pt}
\renewcommand\footrulewidth{0.4pt}

\setlength\parindent{0pt}

%
% Create Problem Sections
%

\newcommand{\enterProblemHeader}[1]{
    \nobreak\extramarks{}{Problem \arabic{#1} continued on next page\ldots}\nobreak{}
    \nobreak\extramarks{Problem \arabic{#1} (continued)}{Problem \arabic{#1} continued on next page\ldots}\nobreak{}
}

\newcommand{\exitProblemHeader}[1]{
    \nobreak\extramarks{Problem \arabic{#1} (continued)}{Problem \arabic{#1} continued on next page\ldots}\nobreak{}
    \stepcounter{#1}
    \nobreak\extramarks{Problem \arabic{#1}}{}\nobreak{}
}

\setcounter{secnumdepth}{0}
\newcounter{partCounter}
\newcounter{homeworkProblemCounter}
\setcounter{homeworkProblemCounter}{1}
\nobreak\extramarks{Problem \arabic{homeworkProblemCounter}}{}\nobreak{}

%
% Homework Problem Environment
%
% This environment takes an optional argument. When given, it will adjust the
% problem counter. This is useful for when the problems given for your
% assignment aren't sequential. See the last 3 problems of this template for an
% example.
%
\newenvironment{homeworkProblem}[1][-1]{
    \ifnum#1>0
        \setcounter{homeworkProblemCounter}{#1}
    \fi
    \section{Problem \arabic{homeworkProblemCounter}}
    \setcounter{partCounter}{1}
    \enterProblemHeader{homeworkProblemCounter}
}{
    \exitProblemHeader{homeworkProblemCounter}
}

%
% Homework Details
%   - Title
%   - Due date
%   - Class
%   - Section/Time
%   - Instructor
%   - Author
%

\newcommand{\hmwkTitle}{Homework\ \#15}
%\newcommand{\hmwkDueDate}{September 17, 2015}
\newcommand{\hmwkClass}{Graph Theory}
\newcommand{\hmwkClassTime}{}
\newcommand{\hmwkClassInstructor}{}
\newcommand{\hmwkAuthorName}{Lin Hung Cheng B01902059}

%
% Title Page

\title{
    \vspace{2in}
    \textmd{\textbf{\hmwkClass:\ \hmwkTitle}}\\
    %\normalsize\vspace{0.1in}\small{Due\ on\ \hmwkDueDate\ at 3:10pm}\\
    %\vspace{0.1in}\large{\textit{\hmwkClassInstructor\ \hmwkClassTime}}
    \vspace{3in}
}

\author{\textbf{\hmwkAuthorName}}
\date{}

\renewcommand{\part}[1]{\textbf{\large Part \Alph{partCounter}}\stepcounter{partCounter}\\}

%
% Various Helper Commands
%

% Useful for algorithms
\newcommand{\alg}[1]{\textsc{\bfseries \footnotesize #1}}

% For derivatives
\newcommand{\deriv}[1]{\frac{\mathrm{d}}{\mathrm{d}x} (#1)}

% For partial derivatives
\newcommand{\pderiv}[2]{\frac{\partial}{\partial #1} (#2)}

% Integral dx
\newcommand{\dx}{\mathrm{d}x}

% Alias for the Solution section header
\newcommand{\solution}{\textbf{\large Solution}}

% Probability commands: Expectation, Variance, Covariance, Bias
\newcommand{\E}{\mathrm{E}}
\newcommand{\Var}{\mathrm{Var}}
\newcommand{\Cov}{\mathrm{Cov}}
\newcommand{\Bias}{\mathrm{Bias}}

\begin{document}

\maketitle

\pagebreak

\begin{homeworkProblem}
  \begin{CJK}{UTF8}{bsmi} % 開始 CJK1

    \textbf{1.}

    令n$\geq$m,令n部份的點為$n_1$, $n_2$,... $n_i$。$n_i$的連邊所著的顏色表示為$\{$$n_i$$m_1$, $n_i$$m_2$, ... $n_i$$m_m$$\}$(為方便,都表示成n=m的形式,在$n>m$的情況下,只需減少$n_i$的長度即可),著色值由1開始計算\\
    在n = 1時,$n_1$ = $\{$1$\}$\\
    在n = 2時,$n_1$ = $\{$1, 2$\}$,$n_2$ = $\{$2, 1$\}$\\
    在n = 3時,$n_1$ =$\{$1, 2, 3$\}$,$n_2$ = $\{$3, 1, 2$\}$,$n_3$ = $\{$2, 3, 1$\}$\\
    在n = 4時,$n_1$ =$\{$1, 2, 3, 4$\}$,$n_2$ = $\{$4, 1, 2, 3$\}$,$n_3$ = $\{$2, 3, 4, 1$\}$, $n_4$ = $\{$3, 4, 1, 2$\}$\\
    在n = 5時,$n_1$ =$\{$1, 2, 3, 4, 5$\}$,$n_2$ = $\{$2, 3, 4, 5, 1$\}$,$n_3$ = $\{$3, 4, 5, 1, 2$\}$, $n_4$ = $\{$4, 5, 1, 2, 3$\}$, $n_5$ = $\{$5, 1, 2, 3, 4, 5$\}$\\
    在n = 6時,$n_1$ =$\{$1, 2, 3, 4, 5, 6$\}$,$n_2$ = $\{$2, 3, 4, 5, 6, 1$\}$,$n_3$ = $\{$3, 4, 5, 6, 1, 2$\}$, $n_4$ = $\{$4, 5, 6, 1, 2, 3$\}$, $n_5$ = $\{$5, 6, 1, 2, 3, 4$\}$, $n_6$ = $\{$6, 1, 2, 3, 4, 5$\}$\\
    以此類推,可得證。

    \textbf{2.}
    
    $Q_1$:明顯成立\\
    $Q_n$(n$\geq$2): 將其拆分成二分圖,由(1)可知二分圖的邊著色數等於最大度數,其邊著色數為n,成立。\\
    
\end{CJK} % 結束 CJK 環境 
\end{homeworkProblem}

\begin{homeworkProblem}
  \begin{CJK}{UTF8}{bsmi} % 開始 CJK2

    設G的邊著色數為最大度數,且G有截點v,G-v的連通部份為$C_1$, $C_2$, ...,其相接點分別為$v_1$, $v_2$, $v_3$...。\\
    因為邊著色數為最大度數,令連到截點的邊的顏色為a,任何連通部份顏色a之外,任何一色的邊均形成完美匹配,其點個數為偶數。\\
    考慮a的完美匹配,可發現$C_i-v_i$為顏色a的完美匹配,所以$C_i-v_i$有偶數個點,與上述條件矛盾。

    %\footnote{https://math.dartmouth.edu/archive/m38s04/public_html/7126.pdf}
    
\end{CJK} % 結束 CJK 環境 
\end{homeworkProblem}

\pagebreak

\begin{homeworkProblem}
  \begin{CJK}{UTF8}{bsmi} % 開始 CJK3
    \solution

    \textbf{(a)}
    
    設G沒有一種2-邊著色使(a)條件成立。\\
    則此時至少有一點v的度數$\geq$2且其相連邊的顏色都相同。\\
    若圖G除去v及其鄰邊後產生兩個以上連通部份,則將其中一個連通部份和其連至v的邊換成另一顏色即可。\\
    若圖G除去v及其鄰邊後產生一個連通部份:\\
    (1)若$v1, v2, ... , vn$其中一點的度數$>$2且除了連至v的邊的c(v)=2,\\
    或是度數=2,均可將連至v的邊換色使條件成立。\\
    (2)若$v1, v2, ... , vn$所有點的度數$>$2且除了連至v的邊的c($v_i$)=1,則可以找到G中包含v的偶圈,令其長度為2m,則在偶圈中和v相距最遠的點w,其在偶圈中的兩邊顏色相同。將其中一邊換色即可使c(v) = 2。\\
    若除了v之外,尚有其他點不合條件,則使用同方法換色即可。

    % http://u.cs.biu.ac.il/~louzouy/courses/graphs/chap6.pdf

    \textbf{(b)}
    
    設H沒有奇圈,則將H用(a)的方法重新著色,可得比f更高的c(v)和,矛盾,所以H有奇圈。

    \textbf{(c)}

    若著色數大於最大度數,設f是$\Delta$-最佳著色,則必有一點u使顏色a出現至少兩次,顏色b沒有出現(因為著色數$>\Delta$)。\\
    由(b)可知,G中存在奇圈,與二分圖矛盾,得證。
    
  \end{CJK} % 結束 CJK 環境 
\end{homeworkProblem}

\begin{homeworkProblem}
  \begin{CJK}{UTF8}{bsmi} % 開始 CJK4

  \end{CJK} % 結束 CJK 環境    
\end{homeworkProblem}

\begin{homeworkProblem}
  \begin{CJK}{UTF8}{bsmi} % 開始 CJK5
    % http://www.iwr.uni-heidelberg.de/groups/comopt/teaching/ws05/graph-theory/graph-theory-4.pdf
    不失一般性,設G為樹;在點個數n=1時,條件成立。\\
    設n=N-1時,條件成立;則n=N時,令x,y是Gn中的兩點。\\
    若xy相連,則移除此邊,G-xy = $G_x$ + $G_y$,且$G_x$和$G_y$符合條件。\\
    令$x'$為x的任一鄰居, $y'$為y的任一鄰居,則$x'$和x,$y'$和y,$y'$和$x'$在$G_3$為鄰居。\\
    此時有一個hamilton圈的路徑為$x-...-x'-y'-...-y-x$。(和G相同處省略)\\
    若xy不相連,則令路徑P為x到y的路徑。取一點x的鄰居z。\\
    令$x'$為x的任一鄰居, $y'$為y的任一鄰居,則$x'$和z在$G_3$為鄰居。\\
    此時有一個hamilton圈的路徑為$x-...-x'-z-...-y'-...-y-x$。(和G相同處省略)\\
    
  \end{CJK} % 結束 CJK 環境    
\end{homeworkProblem}

\end{document}
%%% Local Variables:
%%% mode: latex
%%% TeX-master: t
%%% End:
