\documentclass{article}

\usepackage{fancyhdr}
\usepackage{extramarks}
\usepackage{amsmath}
\usepackage{amsthm}
\usepackage{amsfonts}
\usepackage{tikz}
\usepackage[plain]{algorithm}
\usepackage{algpseudocode}
\usepackage[encapsulated]{CJK}
\usepackage{graphicx}
\usepackage{caption}
\usepackage{subcaption}
\graphicspath{ {./images/} }

\usetikzlibrary{automata,positioning}

%
% Basic Document Settings
%

\topmargin=-0.45in
\evensidemargin=0in
\oddsidemargin=0in
\textwidth=6.5in
\textheight=9.0in
\headsep=0.25in

\linespread{1.1}

\pagestyle{fancy}
\lhead{\hmwkAuthorName}
\chead{\hmwkClass\:\hmwkTitle}
\rhead{\firstxmark}
\lfoot{\lastxmark}
\cfoot{\thepage}

\renewcommand\headrulewidth{0.4pt}
\renewcommand\footrulewidth{0.4pt}

\setlength\parindent{0pt}

%
% Create Problem Sections
%

\newcommand{\enterProblemHeader}[1]{
    \nobreak\extramarks{}{Problem \arabic{#1} continued on next page\ldots}\nobreak{}
    \nobreak\extramarks{Problem \arabic{#1} (continued)}{Problem \arabic{#1} continued on next page\ldots}\nobreak{}
}

\newcommand{\exitProblemHeader}[1]{
    \nobreak\extramarks{Problem \arabic{#1} (continued)}{Problem \arabic{#1} continued on next page\ldots}\nobreak{}
    \stepcounter{#1}
    \nobreak\extramarks{Problem \arabic{#1}}{}\nobreak{}
}

\setcounter{secnumdepth}{0}
\newcounter{partCounter}
\newcounter{homeworkProblemCounter}
\setcounter{homeworkProblemCounter}{1}
\nobreak\extramarks{Problem \arabic{homeworkProblemCounter}}{}\nobreak{}

%
% Homework Problem Environment
%
% This environment takes an optional argument. When given, it will adjust the
% problem counter. This is useful for when the problems given for your
% assignment aren't sequential. See the last 3 problems of this template for an
% example.
%
\newenvironment{homeworkProblem}[1][-1]{
    \ifnum#1>0
        \setcounter{homeworkProblemCounter}{#1}
    \fi
    \section{Problem \arabic{homeworkProblemCounter}}
    \setcounter{partCounter}{1}
    \enterProblemHeader{homeworkProblemCounter}
}{
    \exitProblemHeader{homeworkProblemCounter}
}

%
% Homework Details
%   - Title
%   - Due date
%   - Class
%   - Section/Time
%   - Instructor
%   - Author
%

\newcommand{\hmwkTitle}{Homework\ \#7}
%\newcommand{\hmwkDueDate}{September 17, 2015}
\newcommand{\hmwkClass}{Graph Theory}
\newcommand{\hmwkClassTime}{}
\newcommand{\hmwkClassInstructor}{}
\newcommand{\hmwkAuthorName}{Lin Hung Cheng B01902059}

%
% Title Page


\title{
    \vspace{2in}
    \textmd{\textbf{\hmwkClass:\ \hmwkTitle}}\\
    %\normalsize\vspace{0.1in}\small{Due\ on\ \hmwkDueDate\ at 3:10pm}\\
    %\vspace{0.1in}\large{\textit{\hmwkClassInstructor\ \hmwkClassTime}}
    \vspace{3in}
}

\author{\textbf{\hmwkAuthorName}}
\date{}

\renewcommand{\part}[1]{\textbf{\large Part \Alph{partCounter}}\stepcounter{partCounter}\\}

%
% Various Helper Commands
%

% Useful for algorithms
\newcommand{\alg}[1]{\textsc{\bfseries \footnotesize #1}}

% For derivatives
\newcommand{\deriv}[1]{\frac{\mathrm{d}}{\mathrm{d}x} (#1)}

% For partial derivatives
\newcommand{\pderiv}[2]{\frac{\partial}{\partial #1} (#2)}

% Integral dx
\newcommand{\dx}{\mathrm{d}x}

% Alias for the Solution section header
\newcommand{\solution}{\textbf{\large Solution}}

% Probability commands: Expectation, Variance, Covariance, Bias
\newcommand{\E}{\mathrm{E}}
\newcommand{\Var}{\mathrm{Var}}
\newcommand{\Cov}{\mathrm{Cov}}
\newcommand{\Bias}{\mathrm{Bias}}

\begin{document}

\maketitle

\pagebreak

\begin{homeworkProblem}
  \begin{CJK}{UTF8}{bsmi} % 開始 CJK
定義n維超立方(n-dimensionalhypercube)之點集為所有長度為n 的二進字串,而兩個頂點相鄰若且唯若它們的字串恰有一位不同。如果把頂點的二進字串看作是這些頂點在n 維空間中的座標, 那麼就容易看出Qn 其實就相當於是n 維空間中的單位立方塊, 故名。
    證明對於n ≥ 2, $Q_n$ 至少有$2^{2^{n−2}}$ 種完美匹配。    

    \proof
    
    % http://math.stackexchange.com/questions/687535/hypercube-perfect-matchings
    
    對於n = 2,$Q_n$有$2^{2^{2-2}}$ = 2種完美匹配。\\
    設n = m時,$Q_n$至少有$2^{2^{n-2}}$種完美匹配,則當n =m+1時,可以將$Q_{m+1}$分成2個$Q_m$,每一個$Q_m$都至少有$2^{2^{m-2}}$種完美匹配,在二個$Q_m$不互相匹配的情況下,$Q_{m+1}$至少有$(2^{2^{m-2}})^2 = (2^{2^{m-1}})$種完美匹配。\\
    由歸納法得證,在 n $\geq$ 2, 至少有$2^{2^{n-2}}$種完美匹配
  \end{CJK}
\end{homeworkProblem}

\begin{homeworkProblem}
  \begin{CJK}{UTF8}{bsmi} % 開始 CJK
    4.7 假設二分圖G 的二部份為X 和Y。如果X 中每個點的度數至少為2, 而且|N(S)| ≥ |S| 對所有S ⊆ X 恆成立, 則G 最少有兩個不同的X-完美匹配。\\
    若將定理4.7中的假設「每點的度數至少為2」改為「每點的度數至少為r」, 則可以得到什麼結論
    
    \solution
    
    同定理4.5的第一種證明。\\
    第一種情況,y有r種選法,而對應的r種$G'$皆有完美匹配,因此就得到G至少有r個X-完美匹配。\\
    第二種情況,$G_1$ 有r個$X_1$-完美匹配,其中對x $\in$ S* 皆有d(G1(x)) = d(G(x));而$G_2$至少有一個$X_2$-完美匹配,其中對$x \in X − S*$ 有可能d(G2(x)) $<$ d(G(x)), 所以合起來G至少也有r個X-完美匹配
        
\end{CJK} % 結束 CJK 環境 
\end{homeworkProblem}



\begin{homeworkProblem}
  \begin{CJK}{UTF8}{bsmi} % 開始 CJK
    兩個人在圖G 上玩遊戲, 規則如下: 第一個人先選取任一點開始,之後兩人輪流選取一個未被選過、但和前一次對方選的點相鄰的點。無法繼續選取的人算輸。\\
    試證明:若G 有完美匹配, 則第二個人有必勝策略; 否則第一個人有必勝策略。

    \proof

    令第一個人為$p_1$, 第二個人為$p_2$。\\
    
    若G有完美匹配,則當$p_1$每選一個w, $p_2$選擇w的匹配x,可知只要$p_1$可以選擇一點,$p_2$必可選擇對應的一點,$p_2$必不敗(必勝)。\\
    
    若G無完美匹配,則$p_1$選取不屬於G的最大匹配的點v,之後$p_2$每選一點w,$p_1$選擇w的匹配x。\\
    設$p_2$選一點w,$p_1$無法找到匹配,則選過的點產生的路徑(v ... w)是一個G-可擴展路徑(v和w都不屬於匹配中的點),代表還可以擴張,與無法找到匹配的假設矛盾。\\
    因此,只要$p_2$可以選擇一點,$p_1$必可選擇對應的一點,$p_1$必不敗(必勝)。
    
  \end{CJK} % 結束 CJK 環境 
\end{homeworkProblem}

\pagebreak

\begin{homeworkProblem}
  \begin{CJK}{UTF8}{bsmi} % 開始 CJK
    一個每行每列和都是1 的非負實數矩陣稱為雙重隨機矩陣(doubly stochasticmatrix), 這樣的矩陣中若其元素為0 或1,則稱為置換矩陣(permutationmatrix)。試證明, 任一雙重隨機矩陣Q 都可以表示成Q = c1P1+c2P2+. . .+cmPm, 其中c1, c2, . . . , cm 是和為1 的非負實數, P1, P2, . . . , Pm 為置換矩陣。

    \proof
    %http://math2.uncc.edu/~ghetyei/courses/old/F07.3116/birkhofft.pdf

    令G的行組成的點集為R, doubly stochastic matrix 為 M\\
    設G無完美匹配,不失一般性,設R有一子集r使 $|N(r)| < |r|$ \\
    根據doubly stochastic matrix的性質,屬於r行的元素和應為$|r|$,屬於N(r)列的元素和應為$|N(r)|$;\\
    因為r連接到所有非0的鄰居,$\sum_{i \in r, j \in N(r)} M_{ij} = |r|$。\\
    而N(r)只連接到屬於r的鄰居, $\sum_{i \in r, j \in N(r)} M_{ij} \leq |N(r)|$。\\

    此時 $\sum_{i \in r, j \in N(r)} M_{ij} \leq |N(r)| < |r| = \sum_{i \in r, j \in N(r)} M_{ij}$,矛盾,所以G有完美匹配。

    找到M的完美匹配所對應的元素集S,若其中的最小值為v,則令P為位於S的元素值皆為1的matrix,可找到一個 doubly stochastic matrix $M'$ 使 M = vP + (1-v)$M'$。再尋找$M'$的完美匹配......,直到$M_n$為permutation matrix。
    因為$M'$至少比M少一個非零元素,必能找到$M_n$。\\
    
  \end{CJK} % 結束 CJK 環境    
\end{homeworkProblem}

\begin{homeworkProblem}
  \begin{CJK}{UTF8}{bsmi} % 開始 CJK
    (a) 證明任一圖G 中的一點集S 是獨立集若且唯若S bar是一個點覆蓋, 因此,α(G) + β(G) = |V (G)|。\\
    (b) 證明任一沒有孤立點的圖G 恆有a'(G) + b'(G) = |V (G)|。\\
    (c) 證明若二分圖G 沒有孤立點, 則α(G) =b'(G)\\
    
    \proof
    
    (a)\\
    $\Rightarrow$\\
    設Sbar不是一個點覆蓋,則G中至少有一邊$e=\{x, y\}$ , $x, y \notin V(\bar{s})$, 此時x, y相鄰且 $x, y \in S$, S非獨立集,矛盾。\\
    $\Leftarrow$\\
    設S不是一個獨立集,則S會包含二點x, y 使 $(x, y) \in G(e)$, 此時$\bar{S}$ 不包含x, y, 無法覆蓋邊(x, y), 矛盾。

    max $|S|$ 會產生 min $|\bar{S}|$,即$\alpha$和$\beta$ ,所以$\alpha(G) + \beta(G) = |V(G)|$。\\
    (b)\\
    % http://theory.stanford.edu/~trevisan/cs261/all-notes-2010.pdf
    考慮$a'$(G)以外的點集C,點集中的點互不相鄰,否則$a'$(G)加上兩個相鄰的點會產生更大的matching。\\
    考慮$b'$(G),若使用$a'$(G)+C來覆蓋,則最小值為$|a(G)|+|C| \leq |a(G)|+(|V(G)|-2a(G)) = |V(G)| - a(G)$。

    $a'(G) + b'(G) = a'(G) + (|V(G)|-a'(G)) = |V(G)|$。
    
    (c)\\
    若沒有孤立點,二分圖的minimum edge cover邊數 = minimum vertex cover點數,由(a)可知maximum independent set = minimum vertex cover,在二分圖中也等同於minimum edge cover的邊數,得證。
    
  \end{CJK} % 結束 CJK 環境    
\end{homeworkProblem}

\end{document}

%%% Local Variables:
%%% mode: latex
%%% TeX-master: t
%%% End:
