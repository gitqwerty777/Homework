\documentclass{article}

\usepackage{fancyhdr}
\usepackage{extramarks}
\usepackage{amsmath}
\usepackage{amsthm}
\usepackage{amsfonts}
\usepackage{tikz}
\usepackage[plain]{algorithm}
\usepackage{algpseudocode}
\usepackage[encapsulated]{CJK}
\usepackage{graphicx}
\usepackage{caption}
\usepackage{subcaption}
\graphicspath{ {./images/} }

\usetikzlibrary{automata,positioning}

%
% Basic Document Settings
%

\topmargin=-0.45in
\evensidemargin=0in
\oddsidemargin=0in
\textwidth=6.5in
\textheight=9.0in
\headsep=0.25in

\linespread{1.1}

\pagestyle{fancy}
\lhead{\hmwkAuthorName}
\chead{\hmwkClass\:\hmwkTitle}
\rhead{\firstxmark}
\lfoot{\lastxmark}
\cfoot{\thepage}

\renewcommand\headrulewidth{0.4pt}
\renewcommand\footrulewidth{0.4pt}

\setlength\parindent{0pt}

%
% Create Problem Sections
%

\newcommand{\enterProblemHeader}[1]{
    \nobreak\extramarks{}{Problem \arabic{#1} continued on next page\ldots}\nobreak{}
    \nobreak\extramarks{Problem \arabic{#1} (continued)}{Problem \arabic{#1} continued on next page\ldots}\nobreak{}
}

\newcommand{\exitProblemHeader}[1]{
    \nobreak\extramarks{Problem \arabic{#1} (continued)}{Problem \arabic{#1} continued on next page\ldots}\nobreak{}
    \stepcounter{#1}
    \nobreak\extramarks{Problem \arabic{#1}}{}\nobreak{}
}

\setcounter{secnumdepth}{0}
\newcounter{partCounter}
\newcounter{homeworkProblemCounter}
\setcounter{homeworkProblemCounter}{1}
\nobreak\extramarks{Problem \arabic{homeworkProblemCounter}}{}\nobreak{}

%
% Homework Problem Environment
%
% This environment takes an optional argument. When given, it will adjust the
% problem counter. This is useful for when the problems given for your
% assignment aren't sequential. See the last 3 problems of this template for an
% example.
%
\newenvironment{homeworkProblem}[1][-1]{
    \ifnum#1>0
        \setcounter{homeworkProblemCounter}{#1}
    \fi
    \section{Problem \arabic{homeworkProblemCounter}}
    \setcounter{partCounter}{1}
    \enterProblemHeader{homeworkProblemCounter}
}{
    \exitProblemHeader{homeworkProblemCounter}
}

%
% Homework Details
%   - Title
%   - Due date
%   - Class
%   - Section/Time
%   - Instructor
%   - Author
%

\newcommand{\hmwkTitle}{Homework\ \#6}
%\newcommand{\hmwkDueDate}{September 17, 2015}
\newcommand{\hmwkClass}{Graph Theory}
\newcommand{\hmwkClassTime}{}
\newcommand{\hmwkClassInstructor}{}
\newcommand{\hmwkAuthorName}{Lin Hung Cheng B01902059}

%
% Title Page


\title{
    \vspace{2in}
    \textmd{\textbf{\hmwkClass:\ \hmwkTitle}}\\
    %\normalsize\vspace{0.1in}\small{Due\ on\ \hmwkDueDate\ at 3:10pm}\\
    %\vspace{0.1in}\large{\textit{\hmwkClassInstructor\ \hmwkClassTime}}
    \vspace{3in}
}

\author{\textbf{\hmwkAuthorName}}
\date{}

\renewcommand{\part}[1]{\textbf{\large Part \Alph{partCounter}}\stepcounter{partCounter}\\}

%
% Various Helper Commands
%

% Useful for algorithms
\newcommand{\alg}[1]{\textsc{\bfseries \footnotesize #1}}

% For derivatives
\newcommand{\deriv}[1]{\frac{\mathrm{d}}{\mathrm{d}x} (#1)}

% For partial derivatives
\newcommand{\pderiv}[2]{\frac{\partial}{\partial #1} (#2)}

% Integral dx
\newcommand{\dx}{\mathrm{d}x}

% Alias for the Solution section header
\newcommand{\solution}{\textbf{\large Solution}}

% Probability commands: Expectation, Variance, Covariance, Bias
\newcommand{\E}{\mathrm{E}}
\newcommand{\Var}{\mathrm{Var}}
\newcommand{\Cov}{\mathrm{Cov}}
\newcommand{\Bias}{\mathrm{Bias}}

\begin{document}

\maketitle

\pagebreak

\begin{homeworkProblem}
  \begin{CJK}{UTF8}{bsmi} % 開始 CJK
    針對下面各個問題, 設計出對應的演算法。\\
    (1) 求出一個連通圖中兩點x 和y 的距離。\\
    (2) 判斷一個圖是否為二分圖。
    
    \solution

    \textbf{1.}
    對x作BFS,若找到y,回傳目前深度

    \textbf{2.}

    \begin{algorithm}[]
      \begin{algorithmic}[1]
        \Function{isBipartite}{n}
        \For{each node v in G}
        \If{visit[v] = false}
        \If{visit(v, BLACK) = false}
        \State \Return{false}
        \EndIf
        \EndIf
        \EndFor
        \State \Return{true}
        \EndFunction

        \Function{visit}{v, color}
        \If{visit[v] = true}
        \State \Return{true}
        \EndIf
        \State visit[v] $\gets$ true;
        \State anotherColor $\gets$ (color = BLACK)?WHITE:BLACK;
        \If{color[v] = anotherColor}
        \State \Return{false}
        \EndIf
        \State color[v] $\gets$ color
        \For{each neighbor n of v}
        \If{cannot visit(n, anotherColor)}
        \State \Return{false}
        \EndIf
        \EndFor
        \State \Return{true}
        \EndFunction

      \end{algorithmic}
      \caption{isBipartite}
    \end{algorithm}

  \end{CJK}
\end{homeworkProblem}

\begin{homeworkProblem}
  \begin{CJK}{UTF8}{bsmi} % 開始 CJK
    求圖3.12 中有16 點16 邊的圖的生成樹個數。
    
    \solution
    
    用矩陣-樹定理計算(by matlab)\\
    ans = 2000
   
\end{CJK} % 結束 CJK 環境 
\end{homeworkProblem}

\pagebreak

\begin{homeworkProblem}
  \begin{CJK}{UTF8}{bsmi} % 開始 CJK
    令Gn 是如圖3.13 所示具有2n 點和3n − 2 邊的圖。證明當n ≥ 3 時,
    τ (Gn) = 4τ (Gn−1− τ (Gn−2)。當n ≥ 1 時,求τ (Gn)。

    \proof
    
    從Gn到Gn+1,會多2個點和3個邊,而從$T_n$到$T_{n+1}$,會多2個點和2個邊。\\
    多2個邊的情況有3種,多3個邊、少1個邊的情況有一種。\\
    這四種情況使生成樹數目為$\tau(G_{n+1}) = \tau(G_n) \times 4$,但在少1個邊的情況時,會使$T_n$最後1個方塊的情況減少一種,所以需減掉$\tau(Gn-1)$。\\

  \end{CJK} % 結束 CJK 環境 
\end{homeworkProblem}

\begin{homeworkProblem}
  \begin{CJK}{UTF8}{bsmi} % 開始 CJK
    若T 和T是連通圖G 的兩生成樹, 證明存在e∈ E(T\E(T) 使得T - e + e'和T + e - e'都是G 的生成樹。

    \proof

    令$e = \{x, y\}$,$T'$必存在路徑 x, v1, v2, ..., vk, y,可以找到$e'=(v_i, v_{i+1}) \notin$ T(否則T產生圈)。\\
    此時$T'$ - $e'$ + e無圈(因為去除$e'$,圈仍未產生),且邊數為n-1,為生成樹。\\
    同理可證 T - e + $e'$ 也為生成樹。\\

  \end{CJK} % 結束 CJK 環境    
\end{homeworkProblem}

\begin{homeworkProblem}
  \begin{CJK}{UTF8}{bsmi} % 開始 CJK
    Prim 演算法以下面方法產生一邊賦權連通圖G 的最小生成樹:試證明當演算法結束時, E 是G 的最小生成樹。

    \proof
    
    設$T=\{V, E\}$為G的最小生成樹。使用prim algorithm生成的樹是$T'$:\\
    假設$T \neq T'$:\\
    令$e'$是prim algorithm第一個選擇到不在T的邊,而T選擇e,則$T-e+e'$也是生成樹,因為$e'$必為可連接的邊中權重最小的邊,e的權重w(e)大於等於$e'$的權重$w(e')$。\\
    若$w(e) = w(e')$,則$w(T-e+e')=w(T')$,否則$w(e)>w(e')$使$w(T-e+e')<w(T)$,與T是最小生成樹的假設矛盾。\\
    使用此方法在所有T和$T'$不同的邊,可推得$w(T)=w(T')$。
    
  \end{CJK} % 結束 CJK 環境    
\end{homeworkProblem}

\end{document}

%%% Local Variables:
%%% mode: latex
%%% TeX-master: t
%%% End:
