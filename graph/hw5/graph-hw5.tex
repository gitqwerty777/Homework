\documentclass{article}

\usepackage{fancyhdr}
\usepackage{extramarks}
\usepackage{amsmath}
\usepackage{amsthm}
\usepackage{amsfonts}
\usepackage{tikz}
\usepackage[plain]{algorithm}
\usepackage{algpseudocode}
\usepackage[encapsulated]{CJK}
\usepackage{graphicx}
\usepackage{caption}
\usepackage{subcaption}
\graphicspath{ {./images/} }

\usetikzlibrary{automata,positioning}

%
% Basic Document Settings
%

\topmargin=-0.45in
\evensidemargin=0in
\oddsidemargin=0in
\textwidth=6.5in
\textheight=9.0in
\headsep=0.25in

\linespread{1.1}

\pagestyle{fancy}
\lhead{\hmwkAuthorName}
\chead{\hmwkClass\:\hmwkTitle}
\rhead{\firstxmark}
\lfoot{\lastxmark}
\cfoot{\thepage}

\renewcommand\headrulewidth{0.4pt}
\renewcommand\footrulewidth{0.4pt}

\setlength\parindent{0pt}

%
% Create Problem Sections
%

\newcommand{\enterProblemHeader}[1]{
    \nobreak\extramarks{}{Problem \arabic{#1} continued on next page\ldots}\nobreak{}
    \nobreak\extramarks{Problem \arabic{#1} (continued)}{Problem \arabic{#1} continued on next page\ldots}\nobreak{}
}

\newcommand{\exitProblemHeader}[1]{
    \nobreak\extramarks{Problem \arabic{#1} (continued)}{Problem \arabic{#1} continued on next page\ldots}\nobreak{}
    \stepcounter{#1}
    \nobreak\extramarks{Problem \arabic{#1}}{}\nobreak{}
}

\setcounter{secnumdepth}{0}
\newcounter{partCounter}
\newcounter{homeworkProblemCounter}
\setcounter{homeworkProblemCounter}{1}
\nobreak\extramarks{Problem \arabic{homeworkProblemCounter}}{}\nobreak{}

%
% Homework Problem Environment
%
% This environment takes an optional argument. When given, it will adjust the
% problem counter. This is useful for when the problems given for your
% assignment aren't sequential. See the last 3 problems of this template for an
% example.
%
\newenvironment{homeworkProblem}[1][-1]{
    \ifnum#1>0
        \setcounter{homeworkProblemCounter}{#1}
    \fi
    \section{Problem \arabic{homeworkProblemCounter}}
    \setcounter{partCounter}{1}
    \enterProblemHeader{homeworkProblemCounter}
}{
    \exitProblemHeader{homeworkProblemCounter}
}

%
% Homework Details
%   - Title
%   - Due date
%   - Class
%   - Section/Time
%   - Instructor
%   - Author
%

\newcommand{\hmwkTitle}{Homework\ \#5}
%\newcommand{\hmwkDueDate}{September 17, 2015}
\newcommand{\hmwkClass}{Graph Theory}
\newcommand{\hmwkClassTime}{}
\newcommand{\hmwkClassInstructor}{}
\newcommand{\hmwkAuthorName}{Lin Hung Cheng B01902059}

%
% Title Page


\title{
    \vspace{2in}
    \textmd{\textbf{\hmwkClass:\ \hmwkTitle}}\\
    %\normalsize\vspace{0.1in}\small{Due\ on\ \hmwkDueDate\ at 3:10pm}\\
    %\vspace{0.1in}\large{\textit{\hmwkClassInstructor\ \hmwkClassTime}}
    \vspace{3in}
}

\author{\textbf{\hmwkAuthorName}}
\date{}

\renewcommand{\part}[1]{\textbf{\large Part \Alph{partCounter}}\stepcounter{partCounter}\\}

%
% Various Helper Commands
%

% Useful for algorithms
\newcommand{\alg}[1]{\textsc{\bfseries \footnotesize #1}}

% For derivatives
\newcommand{\deriv}[1]{\frac{\mathrm{d}}{\mathrm{d}x} (#1)}

% For partial derivatives
\newcommand{\pderiv}[2]{\frac{\partial}{\partial #1} (#2)}

% Integral dx
\newcommand{\dx}{\mathrm{d}x}

% Alias for the Solution section header
\newcommand{\solution}{\textbf{\large Solution}}

% Probability commands: Expectation, Variance, Covariance, Bias
\newcommand{\E}{\mathrm{E}}
\newcommand{\Var}{\mathrm{Var}}
\newcommand{\Cov}{\mathrm{Cov}}
\newcommand{\Bias}{\mathrm{Bias}}

\begin{document}

\maketitle

\pagebreak

\begin{homeworkProblem}
  \begin{CJK}{UTF8}{bsmi} % 開始 CJK
    試證明定理3.7 的(5) 與(6) 和其他四個敘述等價。\\
    \solution

    已知(1)(2)(3)(4)等價
    
    (5)$\rightarrow$(1)\\
已知G無圈,設G不連通,可以找到一邊e,其兩點分別屬於二個不同的連通部份,使G+e仍不會產生圈,矛盾。所以G必連通,符合(1)的條件。\\
    (6)$\rightarrow$(1)\\
    已知G連通,設G有圈,可以找到一邊e,為圈的其中一邊,使G-e仍連通,矛盾。所以G必無圈,符合(1)的條件。\\
    (1)$\rightarrow$(5)\\
   樹的定義即包含無圈,設任意加入一條新的邊不會使G有圈,則再任意加入一條邊後的圖G'仍然是樹,可以以此方法加入邊,產生樹$G_1, G_2, G_3$...,使$G_n$產生圈,矛盾。所以任意加入一條新的邊會使G有圈。\\
    (1)$\rightarrow$(6)\\
    樹的定義即包含連通,設任意刪除一條邊不會使G不連通,則刪除一條邊後的圖G'仍是樹,可以以此方法刪除邊,直到圖中沒有任何邊,與連通的假設矛盾。所以任意刪除一條邊會使G不連通。\\
  \end{CJK}
\end{homeworkProblem}

\begin{homeworkProblem}
  \begin{CJK}{UTF8}{bsmi} % 開始 CJK
    若圖G 有n $\geq$ 3 點, 且從G 中去掉任一點均成樹, 試求G 的邊數, 並藉此求G。
    
    \solution
    
    若此時G有n個點,則去掉任一點後的G'會有n-1個點,因為G'是樹,有n-2條邊。\\
    因為去掉任一點之後的邊數相同,可知每個點的度數相同,設其為d。\\
    此時去掉一點會使度數減2d,由總度數變化的式子 dn - 2d = 2(n-2),可求得d = 2。\\
    G會有2n條邊,形成環的形狀。
   
\end{CJK} % 結束 CJK 環境 
\end{homeworkProblem}

\begin{homeworkProblem}
  \begin{CJK}{UTF8}{bsmi} % 開始 CJK
    試證當n ≥ 2 時正整數序列d1, d2, . . . , dn 是某棵樹的度序列之充分必要條件。
    
    \solution
    
    ($\Rightarrow$)\\
    由性質3.6可知n個點的樹有n-1個邊,其度數和即為2(n-1) = 2n-2。\\
    ($\Leftarrow$)\\
    已知度數和為2n-2,在點序列為 $\{v_1 = 1, v_2 = 2,  ... ,v_{n-1} = 2, v_n = 1/}$ 的情況下,必可以找到一棵樹T = {V, E},E為$\{(v_i, v_{i+1}), 1 \leq i \leq n-1\}$。
  \end{CJK} % 結束 CJK 環境 
\end{homeworkProblem}

\begin{homeworkProblem}
  \begin{CJK}{UTF8}{bsmi} % 開始 CJK
    圖G 的中段(median)是指由s(x)最小的所有x所構成的集合。證明樹T的中段恰含一點或恰含相鄰兩點。
    
    \solution
    
    %設樹T的中段為x, y二個不相鄰的點,則可以在x-y路徑之間找到一點z使得\\
    %\begin{equation}
    %  d(z, v)=
    %  \begin{cases}
    %    d(z, v), & \text{if}\ d(z, v) < min(d(x, v)+d(x, z), d(y, v)+d(x, y)) \\
    %    min(d(x, v)+d(x, z), d(y, v)+d(x, y)) , & \text{otherwise}
    %  \end{cases}
    %\end{equation}

    設樹T的中段為x, y二個不相鄰的點,則可以在x-y路徑之間找到如下的z, w。\\
    
    設z為x-y路徑中,最靠近x的點,令x包含z的分支共有a個點,其他分支(不含x)共有b個點,則s(z) = s(x) + b - a\\
    因為$s(x)=s(y)<s(z)$, $b > a$\\

    設w為x-y路徑中,最靠近y的點,令x-y路徑(不含x, y)和x-y路徑中,除了x, y以外的分支共有c個點($0 < c < a$),則s(w) = s(y) + (a-c) - (b+c)\\
    因為$s(x)=s(y)<s(w)$, $a-c > b+c$, $a > b + 2c$\\

    將兩個不等式列出,得$b > a > b+2c$,因為$c > 0$, 矛盾 ;所以x, y必相鄰。

    %s(z') = s(z) + z另(= x另-1) - z(= x+1)
    
    %因為x到y的路徑是唯一的,若d(v, z)的路徑經過x,則d(v, y)也經過x,則d(y, v) > d(v, x) + d(x, z),同理經過y時…。
    %因為所有d(z, v) <= max(d(x, v), d(y, v))所以$\sum{d(z, v)} < \sum{d(x, v)}$,矛盾。
  \end{CJK} % 結束 CJK 環境    
\end{homeworkProblem}

\begin{homeworkProblem}
  \begin{CJK}{UTF8}{bsmi} % 開始 CJK
    設G 是有n 點的連通圖, 定義一個新圖G 其點集為G 的所有生成樹所成的集合, 而兩生成樹相鄰若且唯若它們在G 中有n − 2 條共用邊。證明G是連通圖, 並決定G的直徑。

    \proof
    
    G中任意兩個生成樹T和$T'$,距離為$d \leq n-1$,可以找到在T中一邊$e\notinT'$,和$T'$中一邊$e'\notinT$,如此可以產生一個新樹$T_1$ = T - e + $e'$,$d(T_1, T')$ = d-1,由此方法可以產生$T_2, T_3$...,直到$T_d$使 $ d(T_d, T') = 0$。\\
    由此可知T和T'是連通的,所以G是連通圖。

    % 證每條邊都可換 -> 直徑為n
    因為生成樹每經過一個鄰居可使兩個生成樹的一條邊相同,生成樹最多只有n-1個邊不同,所以只須n-1個鄰居,直徑為n-1。

    %    https://www.math.hmc.edu/~kindred/cuc-only/math104/hmwk-solns/hm3sol.pdf
  \end{CJK} % 結束 CJK 環境    
\end{homeworkProblem}

\end{document}

%%% Local Variables:
%%% mode: latex
%%% TeX-master: t
%%% End:
