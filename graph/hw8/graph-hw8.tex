\documentclass{article}

\usepackage{fancyhdr}
\usepackage{extramarks}
\usepackage{amsmath}
\usepackage{amsthm}
\usepackage{amsfonts}
\usepackage{tikz}
\usepackage[plain]{algorithm}
\usepackage{algpseudocode}
\usepackage[encapsulated]{CJK}
\usepackage{graphicx}
\usepackage{caption}
\usepackage{subcaption}
\graphicspath{ {./images/} }

\usetikzlibrary{automata,positioning}

%
% Basic Document Settings
%

\topmargin=-0.45in
\evensidemargin=0in
\oddsidemargin=0in
\textwidth=6.5in
\textheight=9.0in
\headsep=0.25in

\linespread{1.1}

\pagestyle{fancy}
\lhead{\hmwkAuthorName}
\chead{\hmwkClass\:\hmwkTitle}
\rhead{\firstxmark}
\lfoot{\lastxmark}
\cfoot{\thepage}

\renewcommand\headrulewidth{0.4pt}
\renewcommand\footrulewidth{0.4pt}

\setlength\parindent{0pt}

%
% Create Problem Sections
%

\newcommand{\enterProblemHeader}[1]{
    \nobreak\extramarks{}{Problem \arabic{#1} continued on next page\ldots}\nobreak{}
    \nobreak\extramarks{Problem \arabic{#1} (continued)}{Problem \arabic{#1} continued on next page\ldots}\nobreak{}
}

\newcommand{\exitProblemHeader}[1]{
    \nobreak\extramarks{Problem \arabic{#1} (continued)}{Problem \arabic{#1} continued on next page\ldots}\nobreak{}
    \stepcounter{#1}
    \nobreak\extramarks{Problem \arabic{#1}}{}\nobreak{}
}

\setcounter{secnumdepth}{0}
\newcounter{partCounter}
\newcounter{homeworkProblemCounter}
\setcounter{homeworkProblemCounter}{1}
\nobreak\extramarks{Problem \arabic{homeworkProblemCounter}}{}\nobreak{}

%
% Homework Problem Environment
%
% This environment takes an optional argument. When given, it will adjust the
% problem counter. This is useful for when the problems given for your
% assignment aren't sequential. See the last 3 problems of this template for an
% example.
%
\newenvironment{homeworkProblem}[1][-1]{
    \ifnum#1>0
        \setcounter{homeworkProblemCounter}{#1}
    \fi
    \section{Problem \arabic{homeworkProblemCounter}}
    \setcounter{partCounter}{1}
    \enterProblemHeader{homeworkProblemCounter}
}{
    \exitProblemHeader{homeworkProblemCounter}
}

%
% Homework Details
%   - Title
%   - Due date
%   - Class
%   - Section/Time
%   - Instructor
%   - Author
%

\newcommand{\hmwkTitle}{Homework\ \#8}
%\newcommand{\hmwkDueDate}{September 17, 2015}
\newcommand{\hmwkClass}{Graph Theory}
\newcommand{\hmwkClassTime}{}
\newcommand{\hmwkClassInstructor}{}
\newcommand{\hmwkAuthorName}{Lin Hung Cheng B01902059}

%
% Title Page


\title{
    \vspace{2in}
    \textmd{\textbf{\hmwkClass:\ \hmwkTitle}}\\
    %\normalsize\vspace{0.1in}\small{Due\ on\ \hmwkDueDate\ at 3:10pm}\\
    %\vspace{0.1in}\large{\textit{\hmwkClassInstructor\ \hmwkClassTime}}
    \vspace{3in}
}

\author{\textbf{\hmwkAuthorName}}
\date{}

\renewcommand{\part}[1]{\textbf{\large Part \Alph{partCounter}}\stepcounter{partCounter}\\}

%
% Various Helper Commands
%

% Useful for algorithms
\newcommand{\alg}[1]{\textsc{\bfseries \footnotesize #1}}

% For derivatives
\newcommand{\deriv}[1]{\frac{\mathrm{d}}{\mathrm{d}x} (#1)}

% For partial derivatives
\newcommand{\pderiv}[2]{\frac{\partial}{\partial #1} (#2)}

% Integral dx
\newcommand{\dx}{\mathrm{d}x}

% Alias for the Solution section header
\newcommand{\solution}{\textbf{\large Solution}}

% Probability commands: Expectation, Variance, Covariance, Bias
\newcommand{\E}{\mathrm{E}}
\newcommand{\Var}{\mathrm{Var}}
\newcommand{\Cov}{\mathrm{Cov}}
\newcommand{\Bias}{\mathrm{Bias}}

\begin{document}

\maketitle

\pagebreak

\begin{homeworkProblem}
  \begin{CJK}{UTF8}{bsmi} % 開始 CJK
    有n 個公車司機, n 條費時分別為x1, x2, . . . , xn 的上午路線以及n 條費時分別為y1, y2, . . . ,yn的下午路線。如果他的工時總和超過t就要付他加班費。公司的目標是要分配每個司機一條上午路線及一條下午路線, 使得所有司機超時的總和越小越好。\\
    將上述問題化為加權二分圖問題, 並證明將第i 長的上午路線和第i 短的下午路線分配給同一個司機即可達到公司的目的。\\

    \solution
    
    將上午路線和下午路線視為二分圖的兩邊,形成二分圖,其邊($x_i$, $y_j$)為$\{x_iy_j | x_i+y_j-t \geq 0\}$,權重為$x_i+y_j-t$,其餘邊權重為0,所得的匹配權重和即為加班費總和。\\

    \proof
    
    % 因為時數固定,若能有愈多匹配,將使扣掉的工時總和愈多(若有人工時小於t,代表只扣掉小於t的工時),加班費(= 工時總和 - 扣掉的工時)愈少。\\
    希望匹配得到最低權重。\\
    令 $x_1 > x_2 > x_3 .... > x_n, y_1 < y_2 < y_3 ... < y_n$。\\
    若為匹配 $(x_1, y_1), (x_2, y_2), (x_3, y_3) ... (x_n, y_n)$,則使權重最小($x_1$和$y_1$最不可能有權重$>0$的邊,以此類推)。
    
  \end{CJK}
\end{homeworkProblem}

\begin{homeworkProblem}
  \begin{CJK}{UTF8}{bsmi} % 開始 CJK
    證明一棵樹T 有完美匹配的充要條件是$o(T − v) = 1$ 對T 中的每一個點v 都成立。
    
    \proof\\
    
    \rightarrow\\
    
    樹T有完美匹配,則有一個最大交錯路徑P經過所有點,且點的數目為偶數。\\
    若T-v存在兩個奇連通部份$S_1, S_2$,因為$S_1, S_2$的點在T中也不相連,$S_1, S_2$中的奇點無法匹配,無法達到完美匹配,矛盾。\\
    且T-v有奇數個點,T-v至少有一點奇連通部份。所以T-v恰有一個奇連通部份。\\
    
    \leftarrow\\
    
    % 在n = 2的時候明顯成立。\\
    % 設在 n = m 的時候成立,設v, w為T的leaf,則考慮樹T' = T-v-w中的點x,若T-x只有一個奇連通部份,則T'-x仍然只有一個奇連通部份;若T-x有一個奇連通部份和其他偶連通部份,T'-x,若。\\
    對於每個v,只需找到 (T-v的奇連通部份)$ \cap N(v)$ 的點作匹配即可,因為$o(T-v) = 1$且T為樹,對應的點w恰有一個,若v的奇連通部份為S,此時$w \in S$,則S-w為偶數點,根據條件,不會有2個以上奇連通部份,所以S-w只有偶連通部份。則w的奇連通部份 $\in \bar{S}$,此時$v \in \bar{S}$,所以每兩點會互相對應,可找到一完美匹配。\\
        
\end{CJK} % 結束 CJK 環境 
\end{homeworkProblem}

\pagebreak

\begin{homeworkProblem}
  \begin{CJK}{UTF8}{bsmi} % 開始 CJK
    假設圖G 滿足$o(G − S) ≤ |S|$ 對所有$S ⊆ V (G)$ 皆成立, 而且T 是滿足$o(G − T) = |T|$ 的最大點集。\\
    (a) 證明G − T 的所有連通部分都有奇數點, 進而證明$T = ∅$。\\
    (b) 對於G − T 的任一連通部分C, 證明C − x 對於所有$x ∈ V (G)$ 都滿足Tutte 條件。\\
    (c) 設C 是G − T 的所有連通部分所成的集合, 考慮以$T ∪ C$ 為點集、以${tC : t ∈ T,C ∈ C,NG(t) ∩ C =∅}$為邊集的二分圖H。利用Hall 定理證明H 有一匹配、其大小為|C|。\\
    (d) 利用(a), (b), (c) 證明Tutte 定理。\\

    \proof

    \\
    
    (a)\\
    已知$o(G - T) = |T|$,若G-T存在一連通部分S為偶數點,則可在S中找到一點v,使S-v至少有一個奇數點連通部份。此時T+v為一個更大的點集合,滿足$o(G - T) = |T|$的條件,矛盾。\\
    T若為空集合,此時$o(G) \leq 0$,與G-T的所有連通部分都有奇數點的假設矛盾。所以$ T \neq \emptyset$\\

    (b)\\
    若存在$S \in V(C)$ 使  $o(C-S) > |S|$,則 $T \cup S$ 會使 $o(G-T \cup S ) = (o(G-T) - o(C)) + o(C-S) > |T| - 1 + |S| \neq |T \cup S| = |T| + |S|$,矛盾。\\

    (c)\\
    若對於任何$c \in C$ 都有$|N_{T}(c)| \geq |c|$, 則H中有C-完美匹配,其配對大小為$|C|$。\\
    
    (d)\\
    由(a)可知有 $T \notin \emptyset$,G-T有T個奇連通部份,由(c)得知T可和所有奇連通部份剩下的點相連。其餘點由(b)可知每個C-x都滿足tutte條件,則可遞迴使用此方法匹配;最後所有點被匹配完,可得一完美匹配。\\
    
  \end{CJK} % 結束 CJK 環境 
\end{homeworkProblem}

\begin{homeworkProblem}
  \begin{CJK}{UTF8}{bsmi} % 開始 CJK
    依照下列的男方與女方之偏好順序列表, 求出男方求婚法與女方求婚法的結果。\\
    男方{u, v,w, x, y, z},女方{a, b, c, d, e, f}\\
    $u : a > b > d > c > f > e a : z > x > y > u > v > w\\
    v : a > b > c > f > e > d b : y > z > w > x > v > u\\
    w : c > b > d > a > f > e c : v > x > w > y > u > z\\
    x : c > a > d > b > e > f d : w > y > u > x > z > v\\
    y : c > d > a > b > f > e e : u > v > x > w > y > z\\
    z : d > e > f > c > b > a f : u > w > x > v > z > y\\$
    
    \solution
    
    男:uf vc wb xa yd ze\\
    女:az by cv dw ex fu

  \end{CJK} % 結束 CJK 環境    
\end{homeworkProblem}

\begin{homeworkProblem}
  \begin{CJK}{UTF8}{bsmi} % 開始 CJK
    (a) 證明在求婚法當中, 至多只有一個男生會被拒絕n − 1 次。\\
    (b) 試構造一種偏好組合, 使得進行求婚法的時候每次迴圈過程中都恰只有一個男生被拒絕, 且到最後除了一個男生被拒絕了n − 1 次以外其他男生都被拒絕了n − 2 次。作為推論, 求婚法總會在不超過(n − 1)2 次迴圈內完成。\\

    \solution
    
    (a)設有兩個男生a, b會被拒絕n-1次,令兩人的最後配對為(a, x), (b, y)。\\
    則rank(a, x) = n, rank(b, y) = n, $rank(x, a) < rank(x, b)$, $rank(y, b) < rank(y, a)$。\\
    a 會先向 y 求婚,b 會先向 x 求婚,此時為穩定狀態,矛盾。\\
    若也有最終配對為(c, z)的c向x或y求婚,不失一般性設為x,且 $rank(c, x) < rank(b, x)$,若b向z求婚成功,則(a, y), (b, z), (c, x)求婚會成為最終配對,矛盾;若b向z求婚失敗,則可向當時配對(d,z)的d的最終配對求婚,以此類推,b必可找到尚未配對的人完成最終配對(因為目前有(c-x)配對),矛盾。\\
    若為其他情況,如$rank(b, x) < rank(c, x)$,則不影響(a, y), (b, x)的穩定狀態,仍為矛盾。\\

    所以最多只有一個男生會被拒絕n-1次\\
    
    (b)\\
    n = 3, 男方:${a, b, c}$, 女方:${w, x, y}$\\
    \begin{tabular}{rank}
        & w & x & y \\
      a & 13 & 22 & 31 \\
      b & 12 & 21 & 32 \\
      c & 21 & 13 & 33
    \end{tabular}
    
  \end{CJK} % 結束 CJK 環境    
\end{homeworkProblem}

\end{document}

%%% Local Variables:
%%% mode: latex
%%% TeX-master: t
%%% End:
