\documentclass{article}

\usepackage{fancyhdr}
\usepackage{extramarks}
\usepackage{amsmath}
\usepackage{amsthm}
\usepackage{amsfonts}
\usepackage{tikz}
\usepackage[plain]{algorithm}
\usepackage{algpseudocode}
\usepackage[encapsulated]{CJK}
\usepackage{graphicx}
\usepackage{caption}
\usepackage{subcaption}
\graphicspath{ {./images/} }

\usetikzlibrary{automata,positioning}

%
% Basic Document Settings
%

\topmargin=-0.45in
\evensidemargin=0in
\oddsidemargin=0in
\textwidth=6.5in
\textheight=9.0in
\headsep=0.25in

\linespread{1.1}

\pagestyle{fancy}
\lhead{\hmwkAuthorName}
\chead{\hmwkClass\:\hmwkTitle}
\rhead{\firstxmark}
\lfoot{\lastxmark}
\cfoot{\thepage}

\renewcommand\headrulewidth{0.4pt}
\renewcommand\footrulewidth{0.4pt}

\setlength\parindent{0pt}

%
% Create Problem Sections
%

\newcommand{\enterProblemHeader}[1]{
    \nobreak\extramarks{}{Problem \arabic{#1} continued on next page\ldots}\nobreak{}
    \nobreak\extramarks{Problem \arabic{#1} (continued)}{Problem \arabic{#1} continued on next page\ldots}\nobreak{}
}

\newcommand{\exitProblemHeader}[1]{
    \nobreak\extramarks{Problem \arabic{#1} (continued)}{Problem \arabic{#1} continued on next page\ldots}\nobreak{}
    \stepcounter{#1}
    \nobreak\extramarks{Problem \arabic{#1}}{}\nobreak{}
}

\setcounter{secnumdepth}{0}
\newcounter{partCounter}
\newcounter{homeworkProblemCounter}
\setcounter{homeworkProblemCounter}{1}
\nobreak\extramarks{Problem \arabic{homeworkProblemCounter}}{}\nobreak{}

%
% Homework Problem Environment
%
% This environment takes an optional argument. When given, it will adjust the
% problem counter. This is useful for when the problems given for your
% assignment aren't sequential. See the last 3 problems of this template for an
% example.
%
\newenvironment{homeworkProblem}[1][-1]{
    \ifnum#1>0
        \setcounter{homeworkProblemCounter}{#1}
    \fi
    \section{Problem \arabic{homeworkProblemCounter}}
    \setcounter{partCounter}{1}
    \enterProblemHeader{homeworkProblemCounter}
}{
    \exitProblemHeader{homeworkProblemCounter}
}

%
% Homework Details
%   - Title
%   - Due date
%   - Class
%   - Section/Time
%   - Instructor
%   - Author
%

\newcommand{\hmwkTitle}{Homework\ \#10}
%\newcommand{\hmwkDueDate}{September 17, 2015}
\newcommand{\hmwkClass}{Graph Theory}
\newcommand{\hmwkClassTime}{}
\newcommand{\hmwkClassInstructor}{}
\newcommand{\hmwkAuthorName}{Lin Hung Cheng B01902059}

%
% Title Page


\title{
    \vspace{2in}
    \textmd{\textbf{\hmwkClass:\ \hmwkTitle}}\\
    %\normalsize\vspace{0.1in}\small{Due\ on\ \hmwkDueDate\ at 3:10pm}\\
    %\vspace{0.1in}\large{\textit{\hmwkClassInstructor\ \hmwkClassTime}}
    \vspace{3in}
}

\author{\textbf{\hmwkAuthorName}}
\date{}

\renewcommand{\part}[1]{\textbf{\large Part \Alph{partCounter}}\stepcounter{partCounter}\\}

%
% Various Helper Commands
%

% Useful for algorithms
\newcommand{\alg}[1]{\textsc{\bfseries \footnotesize #1}}

% For derivatives
\newcommand{\deriv}[1]{\frac{\mathrm{d}}{\mathrm{d}x} (#1)}

% For partial derivatives
\newcommand{\pderiv}[2]{\frac{\partial}{\partial #1} (#2)}

% Integral dx
\newcommand{\dx}{\mathrm{d}x}

% Alias for the Solution section header
\newcommand{\solution}{\textbf{\large Solution}}

% Probability commands: Expectation, Variance, Covariance, Bias
\newcommand{\E}{\mathrm{E}}
\newcommand{\Var}{\mathrm{Var}}
\newcommand{\Cov}{\mathrm{Cov}}
\newcommand{\Bias}{\mathrm{Bias}}

\begin{document}

\maketitle

\pagebreak

\begin{homeworkProblem}
  \begin{CJK}{UTF8}{bsmi} % 開始 CJK1

    \solution

    \textbf{1.}
    
    對2-連通圖G進行貪求耳分解,形成$p_1$, $p_2$, $p_3$ ... $p_n$。\\
    只考慮$p_1$時,則依圈的方向組合成$P_3$,明顯條件成立。\\

    考慮$p_1$+$p_2$,先將$p_2$的點排成兩兩相斥的$P_3$,若無多餘點,因為$p_1$, $p_2$都符合條件,條件成立;若有多餘點(最多兩個),則可和$p_1$的點排成$P_3$,剩下的$p_1$仍為$P_N$,組合成$P_3$後,最多只會剩餘兩個點,條件成立。\\
    
    考慮$p_1$+$p_2$+$p_3$,先將$p_3$的點排成兩兩相斥的$P_3$,若$p_3$無多餘的點,條件成立;若$p_3$有多餘點:\\
    
    1. 若$p_3$的端點皆屬於$p_1$:\\
    若$p_1$+$p_2$時,$p_2$沒有多餘點,則將$p_1$+$p_3$用$p_1$+$p_2$的方式組合,條件成立。\\
    否則,$p_2$也有多餘點,因為G是claw-free的圖,此時$p_2$的端點和$p_3$的端點兩兩不同,令$p_2$的端點為$v_1$, $v_2$,$p_3$的端點為$v_3$, $v_4$,\\
    不失一般性,可將$p_1$分為4個點集:b1 = $v_1$~$v_2$(不含$v_1$, $v_2$), $a_1$ = $v_2$~$v_3$, $b_2$ = $v_3$~$v_4$(不含$v_3$, $v_4$), $a_2 = v_4~v_1,|a_1, a_2| \geq 0$, 因為是貪求耳分解,$|p_2| \leq |b_1|$, 且$p_2$有多餘點,$|p_2|>0, |b_1, b_2| > 0$。\\
    先從$p_2$, $p_3$開始組合$P_3$,從$p_2$, $p_3$取$v_1$ = Neighbor(端點)為起始點組合出$P_3$,因為$p_2$, $p_3$各有兩個端點,有2*2 = 4種組合方法,此時$p_1$的兩個連通部分的點集分別為\\
    $\{(a_1), (a_2, b_1, b_2)\}, \{(a_1, b_1), (a_2, b_2)\}, \{(a_1, b_2), (a_2, b_1)\}, \{(a_2), (a_1, b_1, b_2)\}$(若$p_2$, $p_3$只剩一點,需要額外對其中一個連通部分去除一點,以形成$P_3$),經列舉可知,必有一種方法可使兩個連通部分剩餘的點數目 $\leq 2$,條件成立。\\
    
    2. 若$p_3$的端點皆屬於$p_2$,則剩下的$p_2$可能會不連通,令其點集為$r_1$, $r_2$;,在$p_1$以$v_1$和$v_2$為起始/結束點的路徑的點集為$q_1$, $q_2$, 可和$p_2$的點排成$P_3$,$r_1$, $r_2$可和$q_1$, $q_2$排成$P_3$,使$r_1$, $r_2$的點完全用完,由和1.相似的方法列舉可知,必有一種方法可使$q_1$, $q_2$剩餘的點數目$ \leq $2,條件成立。\\

    3. 若$p_3$的端點分別屬於$p_1$, $p_2$,則從$v_1$ = Neighbor(屬於$p_1$的端點)為起始點組合成$P_3$,其餘組合方法與2. 的方法相同。\\

    其餘耳朵也可用此方法遞迴證明條件成立。

   %其實只要證一筆劃問題
    
  \end{CJK}
\end{homeworkProblem}

\begin{homeworkProblem}
  \begin{CJK}{UTF8}{bsmi} % 開始 CJK2

    \proof
    
    區塊可能為一條邊或是2-連通圖。\\
    且因G無偶圈,區塊中無偶圈。\\
    若區塊$G'$為2-連通圖,且無偶圈,此時$G'$的耳分解為$p_1$, $p_2$, ... $p_n$。\\
    $p_1$必為奇圈,若$p_2$的端點為$v_1$, $v_2$:\\
    因為$p_1$為奇圈,$v_1$, $v_2$必有一奇數距離$P_1$,一偶數距離的路徑$P_2$。\\
    若$p_2$上有偶數個邊,則 $P_2$$p_2$ 形成一偶圈;若$p_2$上有奇數個邊,則$v_1$, $P_1$$p_2$ 形成一偶圈;所以$G'$沒有$p_1$以外的耳分解,即$G'$為奇圈。\\
    所以在無偶圈的情況下,區塊只可能為一條邊或是奇圈。
    
\end{CJK} % 結束 CJK 環境 
\end{homeworkProblem}

\pagebreak

\begin{homeworkProblem}
  \begin{CJK}{UTF8}{bsmi} % 開始 CJK3

    \solution
    
    %參考資料:http://web.stanford.edu/class/cme305/Midterm/midterm_soln_2012.pdf\\
    
    不失一般性,可設從X為原點,到$Y=\{0, 1\} \times  k$。\\
    k條路徑的第一條邊分別為(1, 0, ..., 0), (0, 1, ..., 0), (0, 0, ... , 1),分別屬於路徑$p_1$, $p_2$, $p_3$..., $p_k$。\\
    $p_i$的路徑產生方法為:先將第i個座標的值設為1,然後從i+1個座標軸開始,依序將每個座標軸的值修正成和Y相同的值,直到和Y完全相同或是修正到第i個座標軸。\\
    若$Y_i$ = 1,$p_j$(j $\neq$i)必不包含$p_i$經過的邊,因為$p_j$在修正到i時才會使第i個座標軸的值為1,此時已經修正過第j~i座標軸的值。\\
    若$Y_i$ = 0,$p_j$(j $\neq$ i)必不經過$p_i$經過的點,因為$p_j$的第i個座標軸的值永遠為0。\\
    
  \end{CJK} % 結束 CJK 環境 
\end{homeworkProblem}

\begin{homeworkProblem}
  \begin{CJK}{UTF8}{bsmi} % 開始 CJK4

    \solution

    \textbf{1.}
    
    令G為極小2連通圖,此時$\delta(G) \geq \kappa(G) = 2$,而因為G為2-connected,所以有耳分解。\\
    設耳分解的耳朵為$p_1$, $p_2$, $p_3$ ... $p_n$。若$\delta(G) > 2$,則G中的每個點至少都屬於一個耳朵的端點。\\
    此時除去一條邊產生圖$G'$,設邊為$p_i$上的$v_1$$v_2$:\\
    若有$p_j$為$v_1$, $v_2$作為端點,則可將$p_j$放入$p_i$中,取代原本的$v_1$$v_2$,並移除$p_j$;因為此時$p_1$, $p_2$,...$p_i$-$v_1$$v_2$+$p_j$, ... $p_j-1$, $p_j+1$ ...$p_n$為圖$G'$的耳分解;\\
    若無$v_1$, $v_2$作為端點的耳分解,則可找到以$v_1$作端點的$p_j$和以$v_2$作端點的pk,將$p_i$中與$v_1$連通的部分$p_{i1}$放入$p_j$中,將$p_i$中與$v_2$連通的部分$p_{i2}$放入$p_k$中,並移除$p_i$,此時$p_1$, $p_2$,... $p_j$+$p_{i1}$, ... $p_k$+$p_{i2}$ ...$p_n$為圖$G'$的耳分解;\\
    所以$G'$必為2-連通,與G為極小2連通圖的條件矛盾。因為$2 = \kappa(G) \leq \delta(G) \leq 2$,$\delta(G) = 2$。

    \textbf{2.}
    
    令G為極小k邊連通圖,其$\kappa '(G) = k \leq \delta(G)$。\\
    若$\delta(G) > k$,不失一般性,令$\delta(G) = k+1$,則從G中去除一邊,形成圖$G'$,因為...,圖$G'$是k-邊連通,矛盾。
    
    % 證g'是k邊連通

%http://www.math.wvu.edu/~milans/teaching/fa14/math573/hw/hw6.pdf
    
  \end{CJK} % 結束 CJK 環境    
\end{homeworkProblem}

\begin{homeworkProblem}
  \begin{CJK}{UTF8}{bsmi} % 開始 CJK5
    %http://math.stackexchange.com/questions/287063/problem-of-graph-connectivity-with-degree-sequences

    \solution

    \textbf{1.}
    
    令圖G符合k = m時的條件,移除m個點時產生圖$G'$,依所給的條件可知,$G'$中$d_j$的度數 $\geq$ j,$G'$的度序列符合k=0的條件。\\
    %證k=0連通
    若k=0時的圖不連通,不失一般性,設有兩個連通部份A, B, 且A包含$d_n$對應的點, 則A最少有$d_n$+1個點;此時B最多只有n-1-$d_n$個點,因為$d_{n-1-d_n} \geq n-1-d_n$,B的最大點度數的最小值為n-1-$d_n$,與B為一連通部分的假設矛盾,k=0時圖連通。\\
    因為移除m個點後依然連通,k=m時,圖G為m+1連通,得證。

    \textbf{2.}

    $V = \{a, b, c, d\}$\\
    $E = \{ab, ac, ad, bd, cd\}$\\
    j = 1, k = 2\\
    
    若移除k個度數為n-1的點,則G剩下j個度數為j-1的點,n-j-k個度數為n-j-k-1。
    
  \end{CJK} % 結束 CJK 環境    
\end{homeworkProblem}

\end{document}

%%% Local Variables:
%%% mode: latex
%%% TeX-master: t
%%% End:
