\documentclass{article}

\usepackage{fancyhdr}
\usepackage{extramarks}
\usepackage{amsmath}
\usepackage{amsthm}
\usepackage{amsfonts}
\usepackage{tikz}
\usepackage[plain]{algorithm}
\usepackage{algpseudocode}
\usepackage[encapsulated]{CJK}
\usepackage{graphicx}
\usepackage{caption}
\usepackage{subcaption}
\graphicspath{ {./images/} }

\usetikzlibrary{automata,positioning}

%
% Basic Document Settings
%

\topmargin=-0.45in
\evensidemargin=0in
\oddsidemargin=0in
\textwidth=6.5in
\textheight=9.0in
\headsep=0.25in

\linespread{1.1}

\pagestyle{fancy}
\lhead{\hmwkAuthorName}
\chead{\hmwkClass\:\hmwkTitle}
\rhead{\firstxmark}
\lfoot{\lastxmark}
\cfoot{\thepage}

\renewcommand\headrulewidth{0.4pt}
\renewcommand\footrulewidth{0.4pt}

\setlength\parindent{0pt}

%
% Create Problem Sections
%

\newcommand{\enterProblemHeader}[1]{
    \nobreak\extramarks{}{Problem \arabic{#1} continued on next page\ldots}\nobreak{}
    \nobreak\extramarks{Problem \arabic{#1} (continued)}{Problem \arabic{#1} continued on next page\ldots}\nobreak{}
}

\newcommand{\exitProblemHeader}[1]{
    \nobreak\extramarks{Problem \arabic{#1} (continued)}{Problem \arabic{#1} continued on next page\ldots}\nobreak{}
    \stepcounter{#1}
    \nobreak\extramarks{Problem \arabic{#1}}{}\nobreak{}
}

\setcounter{secnumdepth}{0}
\newcounter{partCounter}
\newcounter{homeworkProblemCounter}
\setcounter{homeworkProblemCounter}{1}
\nobreak\extramarks{Problem \arabic{homeworkProblemCounter}}{}\nobreak{}

%
% Homework Problem Environment
%
% This environment takes an optional argument. When given, it will adjust the
% problem counter. This is useful for when the problems given for your
% assignment aren't sequential. See the last 3 problems of this template for an
% example.
%
\newenvironment{homeworkProblem}[1][-1]{
    \ifnum#1>0
        \setcounter{homeworkProblemCounter}{#1}
    \fi
    \section{Problem \arabic{homeworkProblemCounter}}
    \setcounter{partCounter}{1}
    \enterProblemHeader{homeworkProblemCounter}
}{
    \exitProblemHeader{homeworkProblemCounter}
}

%
% Homework Details
%   - Title
%   - Due date
%   - Class
%   - Section/Time
%   - Instructor
%   - Author
%

\newcommand{\hmwkTitle}{Homework\ \#9}
%\newcommand{\hmwkDueDate}{September 17, 2015}
\newcommand{\hmwkClass}{Graph Theory}
\newcommand{\hmwkClassTime}{}
\newcommand{\hmwkClassInstructor}{}
\newcommand{\hmwkAuthorName}{Lin Hung Cheng B01902059}

%
% Title Page


\title{
    \vspace{2in}
    \textmd{\textbf{\hmwkClass:\ \hmwkTitle}}\\
    %\normalsize\vspace{0.1in}\small{Due\ on\ \hmwkDueDate\ at 3:10pm}\\
    %\vspace{0.1in}\large{\textit{\hmwkClassInstructor\ \hmwkClassTime}}
    \vspace{3in}
}

\author{\textbf{\hmwkAuthorName}}
\date{}

\renewcommand{\part}[1]{\textbf{\large Part \Alph{partCounter}}\stepcounter{partCounter}\\}

%
% Various Helper Commands
%

% Useful for algorithms
\newcommand{\alg}[1]{\textsc{\bfseries \footnotesize #1}}

% For derivatives
\newcommand{\deriv}[1]{\frac{\mathrm{d}}{\mathrm{d}x} (#1)}

% For partial derivatives
\newcommand{\pderiv}[2]{\frac{\partial}{\partial #1} (#2)}

% Integral dx
\newcommand{\dx}{\mathrm{d}x}

% Alias for the Solution section header
\newcommand{\solution}{\textbf{\large Solution}}

% Probability commands: Expectation, Variance, Covariance, Bias
\newcommand{\E}{\mathrm{E}}
\newcommand{\Var}{\mathrm{Var}}
\newcommand{\Cov}{\mathrm{Cov}}
\newcommand{\Bias}{\mathrm{Bias}}

\begin{document}

\maketitle

\pagebreak

\begin{homeworkProblem}
  \begin{CJK}{UTF8}{bsmi} % 開始 CJK1

    \solution

    \textbf{1.}

    設$K_{m, n}$的m部份為a個點所在的部份,則$|[S, \bar{S}]|$為\\
    a(n-b)+b(m-a)\\

    \textbf{2.}

    若要使$K_{m, n}$不連通,在移除最少邊的情況下,剩下最大的連通部份為$K_{m, n-1}$或$K_{m-1, n}$\\
    令a = m, b = n-1,此時$|[S, \bar{S}]|$ = m\\
    令a = m-1, b = n,此時$|[S, \bar{S}]|$ = n\\
    可知$\kappa'(K_{m, n}) = min\{m, n\}$\\

    \textbf{3.}

    $K_{3, 3}$至少需3條邊才可連通,所以除去7條邊會剩2條邊,無法連通。\\
    根據公式,邊截集的數目為(1)所示,在$ 0 \leq a, b \leq 3$的條件下,無法找到邊為7的邊截集。
    
  \end{CJK}
\end{homeworkProblem}

\begin{homeworkProblem}
  \begin{CJK}{UTF8}{bsmi} % 開始 CJK2

    \proof

    若 $\delta(G) = n-1$,此時G為完全圖,$\kappa(G) = n-1$\\
    若 $\delta(G) = n-2$,刪除任意n-3個點產生$G'$,此時$G'$仍為連通,因為每個點最多只會和一個點不相連,而$G'$有三個點。\\
    因為$\kappa(G)<\delta(G)$,所以$\kappa(G) = n-2$,得證。\\

    $V = \{a, b, c, d, e\}, E = \{ab, ac, bc, cd, ce, de\}$\\
    $\delta(G) = 2$\\
    $\kappa(G) = 1$(c)\\
    
\end{CJK} % 結束 CJK 環境 
\end{homeworkProblem}

\pagebreak

\begin{homeworkProblem}
  \begin{CJK}{UTF8}{bsmi} % 開始 CJK3

    \solution

    與5.6的證明類似\\
    令$S = \kappa(G)$,$H_1, H_2$ 為 G-S 的兩個連通部份,對於任何$v\in S$,v在$H_1$與$H_2$當中都必各有鄰居。\\
    又因為$\Delta(G) \leq 3$,考慮下列情況:\\
    若deg(v) = 3:\\
    如果v在其中某一部份當中只有一個鄰居,那我們就把連往該鄰居的邊加入切斷集中。\\
    如果v在兩部份當中都只有一個鄰居但第三個鄰居不屬於S,那隨便將連往$H_1$或$H_2$的其中一邊加入切斷集。\\
    如果v在兩部份當中都只有一個鄰居,但第三個鄰居u也屬於S,將v和u連往同一側的邊選出。\\
    若deg(v) = 2,v各有一條邊連到$H_1$, $H_2$,隨便將連往$H_1$或$H_2$的其中一邊加入切斷集。\\
    若deg(v) = 1,則移除此邊即可。\\

    共使用了$|S|$條邊加入切斷集中,得證。
    
  \end{CJK} % 結束 CJK 環境 
\end{homeworkProblem}

\begin{homeworkProblem}
  \begin{CJK}{UTF8}{bsmi} % 開始 CJK4

    \solution
    
    對於只有一個圈的仙人掌,n個點會有n個邊,符合條件。\\
    對於只有圈的仙人掌,可從原本第一個圈的點連出新的圈,每增加一個m個點的圈,會使n增加m-1,邊數增加m。\\
    新增加的點數m-1, 邊數為 m,$3((m-1)-1)/2 \geq m$,化簡得 $m \geq 3$,因為m形成圈,條件必成立。且因為第一個圈已符合條件,整體也符合條件。\\
    對於有邊和圈的仙人掌,每加入一條邊(連結兩個區塊的圈),會增加一個點和一條邊,所以必符合條件,得證。
    
  \end{CJK} % 結束 CJK 環境    
\end{homeworkProblem}

\begin{homeworkProblem}
  \begin{CJK}{UTF8}{bsmi} % 開始 CJK5

    \solution
    
    $\Rightarrow$

    % http://math.stackexchange.com/questions/179759/suppose-g-is-2-connected-show-that-there-exists-a-path-from-x-to-y-cont
    因為G為2連通,y-z必在一圈C上,令在C中的y-z路徑為p,x-y的一條路徑為P:\\
    若$P \cap C = y$,則路徑為P-p(重複的y不計)。\\
    若$|P \cap C|>1$,則找P中第一個和p重複的點v,從v沿著C中的路徑到達y,即可走p。路徑為x-v-y-p(重複的y不計)。\\

    $\Leftarrow$
    
    設去除一點y後,存在x, z使G-y沒有x-z路徑,則代表所有$v \in G-x-y-z$所產生的x-v-z路徑都有經過y,所以在G中找不到能產生x-z-y的路徑,因為x-z路徑中必包含y,矛盾。
    
   
  \end{CJK} % 結束 CJK 環境    
\end{homeworkProblem}

\end{document}

%%% Local Variables:
%%% mode: latex
%%% TeX-master: t
%%% End:
