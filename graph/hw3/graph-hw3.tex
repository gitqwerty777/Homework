\documentclass{article}

\usepackage{fancyhdr}
\usepackage{extramarks}
\usepackage{amsmath}
\usepackage{amsthm}
\usepackage{amsfonts}
\usepackage{tikz}
\usepackage[plain]{algorithm}
\usepackage{algpseudocode}
\usepackage[encapsulated]{CJK}
\usepackage{graphicx}
\usepackage{caption}
\usepackage{subcaption}
\graphicspath{ {./images/} }

\usetikzlibrary{automata,positioning}

%
% Basic Document Settings
%

\topmargin=-0.45in
\evensidemargin=0in
\oddsidemargin=0in
\textwidth=6.5in
\textheight=9.0in
\headsep=0.25in

\linespread{1.1}

\pagestyle{fancy}
\lhead{\hmwkAuthorName}
\chead{\hmwkClass\:\hmwkTitle}
\rhead{\firstxmark}
\lfoot{\lastxmark}
\cfoot{\thepage}

\renewcommand\headrulewidth{0.4pt}
\renewcommand\footrulewidth{0.4pt}

\setlength\parindent{0pt}

%
% Create Problem Sections
%

\newcommand{\enterProblemHeader}[1]{
    \nobreak\extramarks{}{Problem \arabic{#1} continued on next page\ldots}\nobreak{}
    \nobreak\extramarks{Problem \arabic{#1} (continued)}{Problem \arabic{#1} continued on next page\ldots}\nobreak{}
}

\newcommand{\exitProblemHeader}[1]{
    \nobreak\extramarks{Problem \arabic{#1} (continued)}{Problem \arabic{#1} continued on next page\ldots}\nobreak{}
    \stepcounter{#1}
    \nobreak\extramarks{Problem \arabic{#1}}{}\nobreak{}
}

\setcounter{secnumdepth}{0}
\newcounter{partCounter}
\newcounter{homeworkProblemCounter}
\setcounter{homeworkProblemCounter}{1}
\nobreak\extramarks{Problem \arabic{homeworkProblemCounter}}{}\nobreak{}

%
% Homework Problem Environment
%
% This environment takes an optional argument. When given, it will adjust the
% problem counter. This is useful for when the problems given for your
% assignment aren't sequential. See the last 3 problems of this template for an
% example.
%
\newenvironment{homeworkProblem}[1][-1]{
    \ifnum#1>0
        \setcounter{homeworkProblemCounter}{#1}
    \fi
    \section{Problem \arabic{homeworkProblemCounter}}
    \setcounter{partCounter}{1}
    \enterProblemHeader{homeworkProblemCounter}
}{
    \exitProblemHeader{homeworkProblemCounter}
}

%
% Homework Details
%   - Title
%   - Due date
%   - Class
%   - Section/Time
%   - Instructor
%   - Author
%

\newcommand{\hmwkTitle}{Homework\ \#3}
%\newcommand{\hmwkDueDate}{September 17, 2015}
\newcommand{\hmwkClass}{Graph Theory}
\newcommand{\hmwkClassTime}{}
\newcommand{\hmwkClassInstructor}{}
\newcommand{\hmwkAuthorName}{Lin Hung Cheng B01902059}

%
% Title Page
%

\title{
    \vspace{2in}
    \textmd{\textbf{\hmwkClass:\ \hmwkTitle}}\\
    %\normalsize\vspace{0.1in}\small{Due\ on\ \hmwkDueDate\ at 3:10pm}\\
    %\vspace{0.1in}\large{\textit{\hmwkClassInstructor\ \hmwkClassTime}}
    \vspace{3in}
}

\author{\textbf{\hmwkAuthorName}}
\date{}

\renewcommand{\part}[1]{\textbf{\large Part \Alph{partCounter}}\stepcounter{partCounter}\\}

%
% Various Helper Commands
%

% Useful for algorithms
\newcommand{\alg}[1]{\textsc{\bfseries \footnotesize #1}}

% For derivatives
\newcommand{\deriv}[1]{\frac{\mathrm{d}}{\mathrm{d}x} (#1)}

% For partial derivatives
\newcommand{\pderiv}[2]{\frac{\partial}{\partial #1} (#2)}

% Integral dx
\newcommand{\dx}{\mathrm{d}x}

% Alias for the Solution section header
\newcommand{\solution}{\textbf{\large Solution}}

% Probability commands: Expectation, Variance, Covariance, Bias
\newcommand{\E}{\mathrm{E}}
\newcommand{\Var}{\mathrm{Var}}
\newcommand{\Cov}{\mathrm{Cov}}
\newcommand{\Bias}{\mathrm{Bias}}

\begin{document}

\maketitle

\pagebreak

\begin{homeworkProblem}
  \begin{CJK}{UTF8}{bsmi} % 開始 CJK
    用gcd(m, n) 表示非負整數m和n的最大公因數。試證:如果m = an+r, 其中a 和r 是整數且$0 \leq r < n$, 則gcd(m, n)=gcd(n, r)。並證明:這可以用來構造出一個用O(logm + log n)個除法的演算法

    \textbf{1.}
    
    令gcd(m, n) = g,則存在x, y 使 m = gx, n = gy, 且x, y互質;此時 r = g(x-ay)。\\
    因為y 和 x-ay互質,所以 $gcd(n, r) = gcd(gy, g(x-ay)) = g$

    \proof 設 y 和 x-ay 有 gcd = $h > 1$, 則 $y = hj, x-ay = hk$。
    此時 $x = h(aj+k), gcd(x, y) \geq h$, 與x, y互質矛盾
    
    \textbf{2.}

    $m = an + r$, $r < \frac{1}{2}m$
    當r = 0或n = 1時,可以找到gcd = n
    每次執行除法可以使m或n至少減少一半,所以至多需要$\log_2n + \log_2m$次除法
    
    \proof 若 $r \leq \frac{1}{2}m$, 則 $a_n < \frac{1}{2}m$, 且 $n < \frac{1}{2}m$, 與 $n>r$ 矛盾,所以每次執行除法可以使m或n至少減少一半

  \end{CJK}
\end{homeworkProblem}

\begin{homeworkProblem}
  \begin{CJK}{UTF8}{bsmi} % 開始 CJK
    給定實數列$x_1, x_2, . . . , x_n$, 要從中找出一段連續的若干項使其和為最大。試找一個比這個更有效率的演算法。

    \textbf{Solution}

    \begin{algorithm}[]
      \begin{algorithmic}[1]
        \Function{findmaxsequencesum}{$array$}
        \State{} $s[0] \gets 0$
        \State{} $minsum \gets s[0]$
        \State{} $ans \gets - \infty$
        \For{$i = 1, n$}
        \State{} $s[i] \gets s[i-1] + array[i]$
        \State{} $minsum = min(minsum, s[i-1])$
        \If{$s[i]-minsum > ans$}
        \State{} $ans = s[i]-minsum$
        \EndIf
        \EndFor
        \State{} \Return{ans}
        \EndFunction{}
      \end{algorithmic}
      \caption{findMaxSequenceSum}
    \end{algorithm}

  可以達到 $O(n)$ 的時間複雜度

\end{CJK} % 結束 CJK 環境 
\end{homeworkProblem}

\pagebreak

\begin{homeworkProblem}
  \begin{CJK}{UTF8}{bsmi} % 開始 CJK
    雙單連表列A的資料存在陣列DATA中、而其指標維陣列是NEXT 和PREV。(1)請寫出程式,將新資料x 加入A 中、放在i後面。(2) 請寫出程式,將A 中的一項i 刪除。

    \textbf{1.}
    
    \begin{algorithm}[]
      \begin{algorithmic}[1]
        \While{$A.value \neq i$}
        \State{} $A \gets (A \rightarrow next)$
        \EndWhile
        \State{} $n \gets (A \rightarrow next)$
        \State{} $(A \rightarrow next) \gets x$
        \State{} $(n \rightarrow prev) \gets x$
        \State{} $(x \rightarrow next) \gets n$
        \State{} $(x \rightarrow prev) \gets A$
      \end{algorithmic}
      \caption{pushback}
    \end{algorithm}

    \textbf{2.}

    \begin{algorithm}[]
      \begin{algorithmic}[1]
        \While{$A.value \neq i$}
        \State{} $A  \gets  (A \rightarrow next)$
        \EndWhile
        \State{} $p  \gets  (A \rightarrow prev)$
        \State{} $n  \gets  (A \rightarrow next)$
        \State{} $(p  \rightarrow next)  \gets  n$
        \State{} $(n  \rightarrow prev)  \gets  p$
        \State{} $delete A$
      \end{algorithmic}
      \caption{deleteElement}
    \end{algorithm}
    
  \end{CJK} % 結束 CJK 環境 
\end{homeworkProblem}

\begin{homeworkProblem}
  \begin{CJK}{UTF8}{bsmi} % 開始 CJK
    假設圖G = (V,E) 有n 個頂點v1, v2, . . . , vn 及m 條邊e1, e2, . . . , em, 定義G的相連矩陣(incidence matrix) M = [bij ] 是一個n × m 矩陣, 其中bij = 1, 若vi 和ej 相連;0, 若vi 和ej 不相連。試證: MMT = A + 度數矩陣

    \solution

    $MM^T_{ij} = b_{i1}b_{j1} * b_{i2}b_{j2} * b_{i3}b_{j3} ... * b_{im}b_{jm}$\\
    若 i $\neq$ j,其意義為$v_i$和$v_j$是否有在$e_1, e_2, ... e_m$相連,即為$v_i$和$v_j$是否有相連,也就是相鄰矩陣的定義。\\
    若 i = j,則$MM^T_{ij} = MM^T_{ii} = b_{11}b_{11} + b_{22}b_{22} + ... b_{nn}b_{nn} = b_{11} + b_{22} + ... b_{nn} = v_n$的度數。
    $MM^T_{ij}$由此可知,等於相鄰矩陣和度數矩陣的和。

  \end{CJK} % 結束 CJK 環境 
\end{homeworkProblem}

\begin{homeworkProblem}
  \begin{CJK}{UTF8}{bsmi} % 開始 CJK
    假設圖G = (V,E) 中V = {1, 2, . . . , n} 而E 則以m 個有序對表示。我們可以用連續空間(sequential space)來儲存G 中各點的鄰居,也就是在陣列DATA中先存N(1)、然後接續著存N(2)、. . . 依此類推直到放完N(n) 為止; 我們以bi 表示資料N(i) 在DATA 中的起始位置, 於是N(i) 所在的範圍就是從DATA[bi] 到DATA[bi+1 − 1]這一段(其中定義bn+1 為資料結束後的下一個位置)。bi 的值存放在陣列BEG 內。例如, 設n = 9, 而E = {31, 41, 59, 26, 53, 58, 97, 93, 23, 84},則此時DATA 和BEG的資料分別如下:試寫一個演算法(或程式),能讓使用者輸入n 和E, 產生出對應的DATA 和BEG。\\

    \solution

    \begin{algorithm}[]
      \begin{algorithmic}[1]
        \Function{generateDATAandBEG}{n, E}
        \State{} $v \gets array(list())$
        \State{} $DATA \gets array(E.length*2)$ 
        \State{} $BEG \gets array(E.length*2)$
        \For{edge in E}
        \State{} $v[edge.first].push(edge.second)$ \Comment{first and second are two vertex in the edge}
        \State{} $v[edge.second].push(edge.first)$
        \EndFor
        \State{} $BEG[i] \gets 1$
        \For{$i \gets 2, n$}
        \State{} $BEG[i] \gets BEG[i-1]+v[i-1].length$
        \EndFor
        \State{} $index \gets 1$
        \For{$i \gets 1, n$}
        \For{each node N in v[i]}
        \State{} $DATA[index] \gets N$
        \State{} $index \gets index + 1$
        \EndFor
        \EndFor
        \State{} \Return{BEG, DATA}
        \EndFunction
      \end{algorithmic}
      \caption{getBEGandDATA}
    \end{algorithm}

\end{CJK} % 結束 CJK 環境 
\end{homeworkProblem}

\end{document}

%%% Local Variables:
%%% mode: latex
%%% TeX-master: t
%%% End:
