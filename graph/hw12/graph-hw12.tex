\documentclass{article}

\usepackage{fancyhdr}
\usepackage{extramarks}
\usepackage{amsmath}
\usepackage{amsthm}
\usepackage{amsfonts}
\usepackage{tikz}
\usepackage[plain]{algorithm}
\usepackage{algpseudocode}
\usepackage[encapsulated]{CJK}
\usepackage{graphicx}
\usepackage{caption}
\usepackage{subcaption}
\graphicspath{ {./images/} }

\usetikzlibrary{automata,positioning}

%
% Basic Document Settings
%

\topmargin=-0.45in
\evensidemargin=0in
\oddsidemargin=0in
\textwidth=6.5in
\textheight=9.0in
\headsep=0.25in

\linespread{1.1}

\pagestyle{fancy}
\lhead{\hmwkAuthorName}
\chead{\hmwkClass\:\hmwkTitle}
\rhead{\firstxmark}
\lfoot{\lastxmark}
\cfoot{\thepage}

\renewcommand\headrulewidth{0.4pt}
\renewcommand\footrulewidth{0.4pt}

\setlength\parindent{0pt}

%
% Create Problem Sections
%

\newcommand{\enterProblemHeader}[1]{
    \nobreak\extramarks{}{Problem \arabic{#1} continued on next page\ldots}\nobreak{}
    \nobreak\extramarks{Problem \arabic{#1} (continued)}{Problem \arabic{#1} continued on next page\ldots}\nobreak{}
}

\newcommand{\exitProblemHeader}[1]{
    \nobreak\extramarks{Problem \arabic{#1} (continued)}{Problem \arabic{#1} continued on next page\ldots}\nobreak{}
    \stepcounter{#1}
    \nobreak\extramarks{Problem \arabic{#1}}{}\nobreak{}
}

\setcounter{secnumdepth}{0}
\newcounter{partCounter}
\newcounter{homeworkProblemCounter}
\setcounter{homeworkProblemCounter}{1}
\nobreak\extramarks{Problem \arabic{homeworkProblemCounter}}{}\nobreak{}

%
% Homework Problem Environment
%
% This environment takes an optional argument. When given, it will adjust the
% problem counter. This is useful for when the problems given for your
% assignment aren't sequential. See the last 3 problems of this template for an
% example.
%
\newenvironment{homeworkProblem}[1][-1]{
    \ifnum#1>0
        \setcounter{homeworkProblemCounter}{#1}
    \fi
    \section{Problem \arabic{homeworkProblemCounter}}
    \setcounter{partCounter}{1}
    \enterProblemHeader{homeworkProblemCounter}
}{
    \exitProblemHeader{homeworkProblemCounter}
}

%
% Homework Details
%   - Title
%   - Due date
%   - Class
%   - Section/Time
%   - Instructor
%   - Author
%

\newcommand{\hmwkTitle}{Homework\ \#12}
%\newcommand{\hmwkDueDate}{September 17, 2015}
\newcommand{\hmwkClass}{Graph Theory}
\newcommand{\hmwkClassTime}{}
\newcommand{\hmwkClassInstructor}{}
\newcommand{\hmwkAuthorName}{Lin Hung Cheng B01902059}

%
% Title Page


\title{
    \vspace{2in}
    \textmd{\textbf{\hmwkClass:\ \hmwkTitle}}\\
    %\normalsize\vspace{0.1in}\small{Due\ on\ \hmwkDueDate\ at 3:10pm}\\
    %\vspace{0.1in}\large{\textit{\hmwkClassInstructor\ \hmwkClassTime}}
    \vspace{3in}
}

\author{\textbf{\hmwkAuthorName}}
\date{}

\renewcommand{\part}[1]{\textbf{\large Part \Alph{partCounter}}\stepcounter{partCounter}\\}

%
% Various Helper Commands
%

% Useful for algorithms
\newcommand{\alg}[1]{\textsc{\bfseries \footnotesize #1}}

% For derivatives
\newcommand{\deriv}[1]{\frac{\mathrm{d}}{\mathrm{d}x} (#1)}

% For partial derivatives
\newcommand{\pderiv}[2]{\frac{\partial}{\partial #1} (#2)}

% Integral dx
\newcommand{\dx}{\mathrm{d}x}

% Alias for the Solution section header
\newcommand{\solution}{\textbf{\large Solution}}

% Probability commands: Expectation, Variance, Covariance, Bias
\newcommand{\E}{\mathrm{E}}
\newcommand{\Var}{\mathrm{Var}}
\newcommand{\Cov}{\mathrm{Cov}}
\newcommand{\Bias}{\mathrm{Bias}}

\begin{document}

\maketitle

\pagebreak

\begin{homeworkProblem}
  \begin{CJK}{UTF8}{bsmi} % 開始 CJK1
    \solution
    將原本的點集中的$\{0, 1, 2\}$視為點$v_1$, $\{3, 4, 5\}$視為點$v_2$, $\{6, 7, 8\}, \{9, 10, 11\}, \{12, 13\}$分別視為點$v_3, v_4, v_5$。\\
    則此五點形成$K_5$,無法畫出平面圖。
    
  \end{CJK}
\end{homeworkProblem}

\begin{homeworkProblem}
  \begin{CJK}{UTF8}{bsmi} % 開始 CJK2
    \proof
    
    從給定的$p_1$, $p_2$, ...投影到球面上,並在球面上畫出圖G,因為圖G為平面圖,必可於球面上畫出。\\
    再投影回平面上即可。
    
\end{CJK} % 結束 CJK 環境 
\end{homeworkProblem}

\pagebreak

\begin{homeworkProblem}
  \begin{CJK}{UTF8}{bsmi} % 開始 CJK3
    \solution

    由Euler公式可知,$t(K_{4, 4})$和$t(K_{5, 5}) \geq 2$。
    
    \textbf{1.}

    $t(K_{4, 4})$ 可分為$K_{2, 4} + K_{2, 4}$。\\
    $t(K_{4, 4}) = 2$

    \textbf{2.}

    $t(K_{5, 5})$ 可分為兩圖(如右)\\
    $t(K_{5, 5}) = 2$
    
  \end{CJK} % 結束 CJK 環境 
\end{homeworkProblem}

\begin{homeworkProblem}
  \begin{CJK}{UTF8}{bsmi} % 開始 CJK4
    \solution

    令$K_{n, n, n}$的三部分為$G_1, G_2, G_3$
    
    \textbf{(a)}

    可將f(n)分為三部分:$G_1-G_2, G_2-G_3, G_1-G_3$。\\
    三部分內的最少交叉數均為$c(K_{n, n})$,\\
    所以$f(n) \geq 3c(K_{n, n})$。
    
    $f(n) >= c(K_{n, 2n}) = n \times 2n - (6n-4)$\\
    $c(K_{n, n}) = n \times n - (4n-4)$

    將三部分的點放置將平面分成三份的三個軸上,因為從任意兩部分各取兩點,最多有一交叉,\\
    可知每部分最多有$(C^n_2)^2$ 個交叉,所以$f(n) \leq 3(n 2)^2$。

    \textbf{(b)}

    去除一點後,每一個交叉會重複計算3次,產生6個$c(K_{3, 2, 1})$和1個$c(K_{3, 3})$\\
    $3c(K_{3, 3, 1}) \geq 6c(K_{3, 2, 1}) + c(K_{3, 3}) = 7,c(K_{3, 3, 1}) > 2$。\\
    而由下圖可知$c(K_{3, 3, 1}) \leq 3$,所以$c(K_{3, 3, 1}) = 3$。\\

    去除一點後,每一個交叉會重複計算4次,產生6個$c(K_{3, 2, 2})$和2個$c(K_{3, 3, 1})$。\\
    $4c(K_{3, 3, 2}) \geq 6c(K_{3, 2, 2}) + 2c(K_{3, 3, 1}) = 18,c(K_{3, 3, 1}) \geq 5$。\\
    而由下圖可知$c(K_{3, 3, 2}) \leq 7$,得證。\\

    去除一點後,每一個交叉會重複計算5次,產生9個$c(K_{3, 3, 2})$。\\
    $5c(K_{3, 3, 3}) \geq 9c(K_{3, 3, 2})$,代入5可得$c(K_{3, 3, 3}) \geq 9$。\\
    而由下圖可知$c(K_{3, 3, 2}) \leq 15$,得證。\\
    \\
    \\
    \\

    \textbf{(c)}

    n=3時成立。\\
    設n=N-1成立,n=N時,f(N-1)為f(N)在每一部分各移除一點,若交叉的點只在其中兩部分,會重覆$N(N-2)^2$次;\\
    若交叉的點包含3部分的點,會重覆$(N-2)(N-1)^2$次,此兩種交叉的數目皆為f(N-1)。\\
    所以$(N-2)(N-1)^2f(N) \geq N^3 \times f(N-1) = N^3 (N-1)^3 (N-2)/6$, \\
    $f(N) \geq N^3(N-1)/6$,得證。
    
    \textbf{(d)}
    將三部分各放置於二維中的三維座標上,且各有接近一半的點分別放在正負軸上(如圖示)\\
    此時有$2^3 \times 3/2$ = 12種組合,每種組合最多有(如x+, y+)$(C^{n/2}_2)^2)$個交叉,\\
    再加上有12種連接至不同軸的交叉(如線(x+, y+),(x+, z+)的交叉)有$(n/2 2)^2 \times n/2 \times n/2$個交叉,共有\\
    $f(n) \leq 12(3/64n^4 + O(n^3)) = 9/16n^4 + O(n^3)$\\
    
    % http://www.math.ntu.edu.tw/~gjchang/courses/2006-02-graph-theory/sec-6-3-solution-Lin-WH.pdf
  \end{CJK} % 結束 CJK 環境    
\end{homeworkProblem}

\begin{homeworkProblem}
  \begin{CJK}{UTF8}{bsmi} % 開始 CJK5

    \solution

    \textbf{(a)}
    m = 6時成立。$m > 6$時,N若m = M-1成立,在m = M的情況下,\\
    去除M部分中的一點, 產生M個$K_{M-1, n}$。\\
    M的每個交叉重複M-2次,$(M-2)c(K_{M, n}) \geq Mc(K_{M-1, n})$\\
    $c(K_{M, n}) \geq (M/(M-2))((M-1)((M-2)/5)(n/2)((n-1)/2)) = M(M-1)/5(n/2)((n-1)/2))$。\\
    由數學歸納法得證。

    \textbf{(b)}
    $c(K_n) \geq c(K_{n/2, n}) \geq 1/80n^4 + O(n^3)$
    
  \end{CJK} % 結束 CJK 環境    
\end{homeworkProblem}

\end{document}

%%% Local Variables:
%%% mode: latex
%%% TeX-master: t
%%% End:
