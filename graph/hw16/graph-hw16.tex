\documentclass{article}

\usepackage{fancyhdr}
\usepackage{extramarks}
\usepackage{amsmath}
\usepackage{amsthm}
\usepackage{amsfonts}
\usepackage{tikz}
\usepackage[plain]{algorithm}
\usepackage{algpseudocode}
\usepackage[encapsulated]{CJK}
\usepackage{graphicx}
\usepackage{caption}
\usepackage{subcaption}
\usepackage{graphics, tikz, tkz-berge, tkz-graph}
\graphicspath{ {./images/} }

\usetikzlibrary{automata,positioning}

%
% Basic Document Settings
%

\topmargin=-0.45in
\evensidemargin=0in
\oddsidemargin=0in
\textwidth=6.5in
\textheight=9.0in
\headsep=0.25in

\linespread{1.1}

\pagestyle{fancy}
\lhead{\hmwkAuthorName}
\chead{\hmwkClass\:\hmwkTitle}
\rhead{\firstxmark}
\lfoot{\lastxmark}
\cfoot{\thepage}

\renewcommand\headrulewidth{0.4pt}
\renewcommand\footrulewidth{0.4pt}

\setlength\parindent{0pt}

%
% Create Problem Sections
%

\newcommand{\enterProblemHeader}[1]{
    \nobreak\extramarks{}{Problem \arabic{#1} continued on next page\ldots}\nobreak{}
    \nobreak\extramarks{Problem \arabic{#1} (continued)}{Problem \arabic{#1} continued on next page\ldots}\nobreak{}
}

\newcommand{\exitProblemHeader}[1]{
    \nobreak\extramarks{Problem \arabic{#1} (continued)}{Problem \arabic{#1} continued on next page\ldots}\nobreak{}
    \stepcounter{#1}
    \nobreak\extramarks{Problem \arabic{#1}}{}\nobreak{}
}

\setcounter{secnumdepth}{0}
\newcounter{partCounter}
\newcounter{homeworkProblemCounter}
\setcounter{homeworkProblemCounter}{1}
\nobreak\extramarks{Problem \arabic{homeworkProblemCounter}}{}\nobreak{}

%
% Homework Problem Environment
%
% This environment takes an optional argument. When given, it will adjust the
% problem counter. This is useful for when the problems given for your
% assignment aren't sequential. See the last 3 problems of this template for an
% example.
%
\newenvironment{homeworkProblem}[1][-1]{
    \ifnum#1>0
        \setcounter{homeworkProblemCounter}{#1}
    \fi
    \section{Problem \arabic{homeworkProblemCounter}}
    \setcounter{partCounter}{1}
    \enterProblemHeader{homeworkProblemCounter}
}{
    \exitProblemHeader{homeworkProblemCounter}
}

%
% Homework Details
%   - Title
%   - Due date
%   - Class
%   - Section/Time
%   - Instructor
%   - Author
%

\newcommand{\hmwkTitle}{Homework\ \#16}
%\newcommand{\hmwkDueDate}{September 17, 2015}
\newcommand{\hmwkClass}{Graph Theory}
\newcommand{\hmwkClassTime}{}
\newcommand{\hmwkClassInstructor}{}
\newcommand{\hmwkAuthorName}{Lin Hung Cheng B01902059}

%
% Title Page

\title{
    \vspace{2in}
    \textmd{\textbf{\hmwkClass:\ \hmwkTitle}}\\
    %\normalsize\vspace{0.1in}\small{Due\ on\ \hmwkDueDate\ at 3:10pm}\\
    %\vspace{0.1in}\large{\textit{\hmwkClassInstructor\ \hmwkClassTime}}
    \vspace{3in}
}

\author{\textbf{\hmwkAuthorName}}
\date{}

\renewcommand{\part}[1]{\textbf{\large Part \Alph{partCounter}}\stepcounter{partCounter}\\}

%
% Various Helper Commands
%

% Useful for algorithms
\newcommand{\alg}[1]{\textsc{\bfseries \footnotesize #1}}

% For derivatives
\newcommand{\deriv}[1]{\frac{\mathrm{d}}{\mathrm{d}x} (#1)}

% For partial derivatives
\newcommand{\pderiv}[2]{\frac{\partial}{\partial #1} (#2)}

% Integral dx
\newcommand{\dx}{\mathrm{d}x}

% Alias for the Solution section header
\newcommand{\solution}{\textbf{\large Solution}}

% Probability commands: Expectation, Variance, Covariance, Bias
\newcommand{\E}{\mathrm{E}}
\newcommand{\Var}{\mathrm{Var}}
\newcommand{\Cov}{\mathrm{Cov}}
\newcommand{\Bias}{\mathrm{Bias}}

\begin{document}

\maketitle

\pagebreak

\begin{homeworkProblem}
  \begin{CJK}{UTF8}{bsmi} % 開始 CJK1

    $(1)\rightarrow(2)$\\
    由(1)可知$deg(u), deg(v) \geq \frac{n}{2}$\\
    ,所以$deg(u) + deg(v) \geq n$

    $(2)\rightarrow(3)$\\
    若(2)成立時,$d_k <= k, 1 \leq k < \frac{n}{2}$\\
    則可以找到$d_i, d_j, ij \notin G, 1 \leq i, j < \frac{n}{2}$使deg(i)+deg(j) $<$ n\\
    (因為$deg(i)< \frac{n}{2}$,i不可能連至所有$d_j\{j < \frac{n}{2}\}$),矛盾。

    $(3)\rightarrow(4)$\\
    由(3)知,若$d_j < j$,則$j \geq \frac{n}{2}$,所以j, k都大於$\frac{n}{2}$,因此$d_j, d_k > d_{\frac{n}{2}}$,$d_j + d_k \geq n$。

    $(4)\rightarrow(5)$\\
    (5)的k, n-k代入(4),可得$k < n-k$,$d_k \leq k$,若$d_{n-k} < n-k$,則$d_k + d_{n-k} \geq n$,$d_{n-k} \geq n-k$,矛盾。所以$d_{n-k} \geq n-k$。

    $(5)\rightarrow(6)$\\
    設符合條件(6)時會使$d_i + d_j < n$\\
    考慮符合(6)條件的$(i, j) = (k, n-k)$(若i+j = n時成立, j更大的情況也會成立),且令$i \leq j$。\\
    則若$i < \frac{n}{2}$,由(5)可知,$d_j \geq n-k$,與(6)的條件矛盾。\\
    所以$i \geq \frac{n}{2}$,此時$d_i < i, d_j < j$,且因為G不包含ij,\\
    ...
    
    $(6)\rightarrow(7)$\\


\end{CJK} % 結束 CJK 環境 
\end{homeworkProblem}

\begin{homeworkProblem}
  \begin{CJK}{UTF8}{bsmi} % 開始 CJK2

    \solution

    取3-邊著色中的兩色為邊的$G'$,此時所有點的度數皆為2,即為一圈,且每一點皆經過,為一hamilton圈。
    
\end{CJK} % 結束 CJK 環境 
\end{homeworkProblem}

\pagebreak

\begin{homeworkProblem}
  \begin{CJK}{UTF8}{bsmi} % 開始 CJK3

    \solution
    
    
  \end{CJK} % 結束 CJK 環境 
\end{homeworkProblem}

\begin{homeworkProblem}
  \begin{CJK}{UTF8}{bsmi} % 開始 CJK4

    \solution
    圖一有1個八邊的面,6個四邊的面。\\
    由Grinberg定理可知,$6(f_8-g_8) + 4(f_6-g_6) = 0$\\
    由於八邊的面為外邊,可寫成$4(f_6-g_6) = 6$。\\
    不成立,所以無hamilton圈。

    圖二有3個四邊的面,6個六邊的面。\\
    由Grinberg定理可知,$4(f_6-g_6) + 2(f_4-g_4) = 0$\\
    可由下圖路徑得到hamilton圈。
    
  \end{CJK} % 結束 CJK 環境    
\end{homeworkProblem}

\begin{homeworkProblem}
  \begin{CJK}{UTF8}{bsmi} % 開始 CJK5

    \proof

    21個五邊的面,3個八邊的面,1個九邊的面。\\
    由Grinberg定理可知,3($f_5-g_5$) + 6($f_8-g_8$) + 7($f_9-g_9$) = 0,而因為九邊的面必為外邊,$f_9-g_9=-1$。\\
    所以3($f_5-g_5$) + 6($f_8-g_8$) = 7,不可能成立。
    
  \end{CJK} % 結束 CJK 環境    
\end{homeworkProblem}

\end{document}
%%% Local Variables:
%%% mode: latex
%%% TeX-master: t
%%% End:
