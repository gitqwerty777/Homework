\documentclass{article}

\usepackage{fancyhdr}
\usepackage{extramarks}
\usepackage{amsmath}
\usepackage{amsthm}
\usepackage{amsfonts}
\usepackage{tikz}
\usepackage[plain]{algorithm}
\usepackage{algpseudocode}
\usepackage[encapsulated]{CJK}
\usepackage{graphicx}
\graphicspath{ {./images/} }

\usetikzlibrary{automata,positioning}

%
% Basic Document Settings
%

\topmargin=-0.45in
\evensidemargin=0in
\oddsidemargin=0in
\textwidth=6.5in
\textheight=9.0in
\headsep=0.25in

\linespread{1.1}

\pagestyle{fancy}
\lhead{\hmwkAuthorName}
\chead{\hmwkClass\:\hmwkTitle}
\rhead{\firstxmark}
\lfoot{\lastxmark}
\cfoot{\thepage}

\renewcommand\headrulewidth{0.4pt}
\renewcommand\footrulewidth{0.4pt}

\setlength\parindent{0pt}

%
% Create Problem Sections
%

\newcommand{\enterProblemHeader}[1]{
    \nobreak\extramarks{}{Problem \arabic{#1} continued on next page\ldots}\nobreak{}
    \nobreak\extramarks{Problem \arabic{#1} (continued)}{Problem \arabic{#1} continued on next page\ldots}\nobreak{}
}

\newcommand{\exitProblemHeader}[1]{
    \nobreak\extramarks{Problem \arabic{#1} (continued)}{Problem \arabic{#1} continued on next page\ldots}\nobreak{}
    \stepcounter{#1}
    \nobreak\extramarks{Problem \arabic{#1}}{}\nobreak{}
}

\setcounter{secnumdepth}{0}
\newcounter{partCounter}
\newcounter{homeworkProblemCounter}
\setcounter{homeworkProblemCounter}{1}
\nobreak\extramarks{Problem \arabic{homeworkProblemCounter}}{}\nobreak{}

%
% Homework Problem Environment
%
% This environment takes an optional argument. When given, it will adjust the
% problem counter. This is useful for when the problems given for your
% assignment aren't sequential. See the last 3 problems of this template for an
% example.
%
\newenvironment{homeworkProblem}[1][-1]{
    \ifnum#1>0
        \setcounter{homeworkProblemCounter}{#1}
    \fi
    \section{Problem \arabic{homeworkProblemCounter}}
    \setcounter{partCounter}{1}
    \enterProblemHeader{homeworkProblemCounter}
}{
    \exitProblemHeader{homeworkProblemCounter}
}

%
% Homework Details
%   - Title
%   - Due date
%   - Class
%   - Section/Time
%   - Instructor
%   - Author
%

\newcommand{\hmwkTitle}{Homework\ \#5}
%\newcommand{\hmwkDueDate}{September 17, 2015}
\newcommand{\hmwkClass}{Information Retrieval}
\newcommand{\hmwkClassTime}{}
\newcommand{\hmwkClassInstructor}{}
\newcommand{\hmwkAuthorName}{Lin Hung Cheng B01902059}

%
% Title Page
%

\title{
    \vspace{2in}
    \textmd{\textbf{\hmwkClass:\ \hmwkTitle}}\\
    %\normalsize\vspace{0.1in}\small{Due\ on\ \hmwkDueDate\ at 3:10pm}\\
    %\vspace{0.1in}\large{\textit{\hmwkClassInstructor\ \hmwkClassTime}}
    \vspace{3in}
}

\author{\textbf{\hmwkAuthorName}}
\date{}

\renewcommand{\part}[1]{\textbf{\large Part \Alph{partCounter}}\stepcounter{partCounter}\\}

%
% Various Helper Commands
%

% Useful for algorithms
\newcommand{\alg}[1]{\textsc{\bfseries \footnotesize #1}}

% For derivatives
\newcommand{\deriv}[1]{\frac{\mathrm{d}}{\mathrm{d}x} (#1)}

% For partial derivatives
\newcommand{\pderiv}[2]{\frac{\partial}{\partial #1} (#2)}

% Integral dx
\newcommand{\dx}{\mathrm{d}x}

% Alias for the Solution section header
\newcommand{\solution}{\textbf{\large Solution}}

% Probability commands: Expectation, Variance, Covariance, Bias
\newcommand{\E}{\mathrm{E}}
\newcommand{\Var}{\mathrm{Var}}
\newcommand{\Cov}{\mathrm{Cov}}
\newcommand{\Bias}{\mathrm{Bias}}

\begin{document}

\maketitle

\pagebreak

\begin{homeworkProblem}
  \begin{CJK}{UTF8}{bsmi} % 開始 CJK 
    \textbf{Solution}

    add collection size into consideration: \\

    new score = $(1-a) \times original_score$(like F1, MAP, which with precision and recall) $ + (a) \times \log{collection_size} $, 
    a is the parameter such that $0 \leq a \leq 1$

    The new score can properly reflect the collection size in evaluation
    
    But I think we should use the same collection to compare evaluation score in the regular basis.
    
  \end{CJK}
\end{homeworkProblem}

\begin{homeworkProblem}
  \begin{CJK}{UTF8}{bsmi} % 開始 CJK

    \textbf{1.}
    
    transform matrix = $0.2e + 0.8A^T$\\
    A = \[ \left(
        \begin{array}{ccccc}
          0 & 0.5 & 0.5 & 0 & 0\\
          0 & 0 & 0 & 0.5 & 0.5\\
          0 & 1 & 0 & 0 & 0\\
          0 & 0 & 1 & 0 & 0\\          
          0 & 0 & 0 & 1 & 0\\
        \end{array} \right)\]

    transform matrix = \[ \left(
        \begin{array}{ccccc}
          0.04 & 0.04 & 0.04 & 0.04 & 0.04\\
          0.44 & 0.04 & 0.84 & 0.04 & 0.04\\
          0.44 & 0.04 & 0.04 & 0.84 & 0.04\\
          0.04 & 0.44 & 0.04 & 0.04 & 0.84\\          
          0.04 & 0.44 & 0.04 & 0.04 & 0.04\\
        \end{array} \right)\] 


    \textbf{2.}
    \[ \left(
        \begin{array}{ccccc}
          0 & 0.3 & 0.3 & 0.275 & 0.125\\
        \end{array} \right)\] 

    code link: http://ppt.cc/CvN9Z

  \end{CJK} % 結束 CJK 環境 
\end{homeworkProblem}

\begin{homeworkProblem}
  \begin{CJK}{UTF8}{bsmi} % 開始 CJK

    \textbf{1.}
    L = \[ \left(
        \begin{array}{ccccc}
          0 & 1 & 1 & 0 & 0\\
          0 & 0 & 0 & 1 & 1\\
          0 & 1 & 0 & 0 & 0\\
          0 & 0 & 1 & 0 & 0\\          
          0 & 0 & 0 & 1 & 0\\
        \end{array}
      \right)\]

    L^TL = \[ \left(
        \begin{array}{ccccc}
          0 & 0 & 0 & 0 & 0\\
          0 & 2 & 1 & 0 & 0\\
          0 & 1 & 2 & 0 & 0\\
          0 & 0 & 0 & 2 & 1\\          
          0 & 0 & 0 & 1 & 1\\
        \end{array}
      \right)\]         

    LL^T = \[ \left(
        \begin{array}{ccccc}
          2 & 0 & 1 & 1 & 0\\
          0 & 2 & 0 & 0 & 1\\
          1 & 0 & 1 & 0 & 0\\
          1 & 0 & 0 & 1 & 0\\          
          0 & 1 & 0 & 0 & 1\\
        \end{array}
      \right)\] 

    

  \end{CJK} % 結束 CJK 環境 
\end{homeworkProblem}

\end{document}

%%% Local Variables:
%%% mode: latex
%%% TeX-master: t
%%% End:
