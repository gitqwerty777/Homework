\documentclass[xcolor=dvipsnames, utf8]{beamer}
\usepackage{listings}
\usepackage{CJKutf8}
\usepackage{graphicx}
\graphicspath{{./figs/}{./}}

\usecolortheme[named=Green]{structure}
\usetheme{Warsaw}
\useoutertheme{infolines}

\begin{document}

\begin{CJK}{UTF8}{bsmi}
\logo{\includegraphics[width=5cm]{emacs-move.png}}
\title{go}
\author{HCL}
\institute{NTU CSIE}
\date{2015/07/09}

\begin{frame}
\titlepage
\end{frame}

\begin{frame}
\frametitle{Beam Theme}
\begin{itemize}
\item<1-> AnnArbor
Antibes
Bergen
Berkeley
Berlin
Boadilla
CambridgeUS
Copenhagen
Dresden
Frankfurt
Goettingen
Hannover
Ilmenau
JuanLesPins
Luebeck
\item<2->Madrid
Marburg
Montpellier
PaloAlto
\only<2->{only2} % looks like the same to uncover
\uncover<2->{uncover2}
Pittsburgh
\item<3->Rochester
Singapore
Szeged
\end{itemize}
\end{frame}

\begin{frame}
\frametitle{t2}
\color<2>{green}{green words only at second page\\} % page specific command should put at first
\color{green}{green words\\}
\alert{alert words}
\end{frame}

\begin{frame}[label=t3]
\frametitle{t3}
\begin{block}{small point}
green title
\end{block}

\begin{alertblock}{big point}
red title
\end{alertblock}

\end{frame}

\begin{frame}[fragile]{t4}
\lstset{language=C++,
                basicstyle=\ttfamily,
                keywordstyle=\color{blue}\ttfamily,
                stringstyle=\color{red}\ttfamily,
                commentstyle=\color{green}\ttfamily,
                morecomment=[l][\color{magenta}]{\#}
}
\begin{lstlisting}
  #include<stdio.h>
  #include<iostream>
  // C++ lstlist example
  int main(void){
    printf("Hello World\n");
    return 0;
  }
\end{lstlisting}

\hyperlink{t3}{\beamerbutton{go to t3}}
\end{frame}

\begin{frame}{第五張投影片}
中文測試 % https://www.ptt.cc/bbs/LaTeX/M.1210092248.A.597.html




\end{frame}

\begin{frame}[fragile]{frame6}
\lstset{language=python,
        numbers=left,
        numberstyle=\tiny,
        showstringspaces=false,
        %aboveskip=40pt,
        frame=leftline
}

\begin{lstlisting}
  import random
  for i in range(10): # loop
  j = random.randint()
  print j
\end{lstlisting}

\end{frame}

\begin{frame}{frame7-insert code file}
  \lstset{language=python, 
    keywordstyle=\color{blue}\ttfamily,
    stringstyle=\color{red}\ttfamily,
    commentstyle=\color{green}\ttfamily,
    morecomment=[l][\color{magenta}]{\#}
  }
  \lstinputlisting{../NLP/p1/include.py}
\end{frame}

\begin{frame}

$formula() = X^1_2$


資料來源:
蔡炎龍
政治大學應用數學系
\end{frame}

\end{CJK}
\end{document}
