\documentclass{article}
\usepackage[encapsulated]{CJK}
\begin{document}
%中文時不能用 \fontsize{12pt}{20pt}\selectfont

\begin{CJK}{UTF8}{bsmi} % 開始 CJK 環境,設定編碼,設定字體
若有N個綠州,分別為 $X_1, X_2 ... X_n $\\
則可令留在最後一個綠州的機率值(和機率成比例)

$ V_n = 1 $\\
其實際機率為P(Stay at $X_{n}$) = $V_n / sumof(V)$\\
因為P(Stay at $X_{n}$) = P(Leave $X_{n-1}$ , 下一個(反面), 留下(正面))\\
且因為第一次丟銅板機率均為$\frac{1}{2}$,P(Stay at $X_{n-1}$) = $V_{n-1}\times sumof(V)$ = P(Leave $X_{n-1}$)\\
可以推導出

$V_{n} = V_{n-1}/4$\\
所以$V_{n-1} = 4 \times V_{n} = 4$\\
由相同技巧推導,可以得知\\
$ V_n = 1 $, $V_{n-1} = 4$\\
$V_{n-2} = 4 \times V_{n-1} - 2 \times V_n$ -- 特殊\\
$V_{n-3} = 4 \times V_{n-2} - V_{n-1}$\\
$V_{n-4} = 4 \times V_{n-3} - V_{n-2}$\\
......\\
$V_{2} = 4 \times V_{3} - V_{4}$\\
$V_{1} = (4\times V_{2}-V_{3})/2$ -- 特殊\\

則留在第k個綠州的機率為

P(Stay at $X_{k}$) = $V_{k}/sumof(V)$
\end{CJK} % 結束 CJK 環境
\end{document}
